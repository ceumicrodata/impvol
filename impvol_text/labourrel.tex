%2multibyte Version: 5.50.0.2960 CodePage: 65001

\documentclass{article}
%%%%%%%%%%%%%%%%%%%%%%%%%%%%%%%%%%%%%%%%%%%%%%%%%%%%%%%%%%%%%%%%%%%%%%%%%%%%%%%%%%%%%%%%%%%%%%%%%%%%%%%%%%%%%%%%%%%%%%%%%%%%%%%%%%%%%%%%%%%%%%%%%%%%%%%%%%%%%%%%%%%%%%%%%%%%%%%%%%%%%%%%%%%%%%%%%%%%%%%%%%%%%%%%%%%%%%%%%%%%%%%%%%%%%%%%%%%%%%%%%%%%%%%%%%%%
%TCIDATA{OutputFilter=LATEX.DLL}
%TCIDATA{Version=5.50.0.2960}
%TCIDATA{Codepage=65001}
%TCIDATA{<META NAME="SaveForMode" CONTENT="1">}
%TCIDATA{BibliographyScheme=Manual}
%TCIDATA{Created=Monday, November 30, 2015 16:55:25}
%TCIDATA{LastRevised=Monday, November 30, 2015 17:05:20}
%TCIDATA{<META NAME="GraphicsSave" CONTENT="32">}
%TCIDATA{<META NAME="DocumentShell" CONTENT="Standard LaTeX\Blank - Standard LaTeX Article">}
%TCIDATA{CSTFile=40 LaTeX article.cst}

\newtheorem{theorem}{Theorem}
\newtheorem{acknowledgement}[theorem]{Acknowledgement}
\newtheorem{algorithm}[theorem]{Algorithm}
\newtheorem{axiom}[theorem]{Axiom}
\newtheorem{case}[theorem]{Case}
\newtheorem{claim}[theorem]{Claim}
\newtheorem{conclusion}[theorem]{Conclusion}
\newtheorem{condition}[theorem]{Condition}
\newtheorem{conjecture}[theorem]{Conjecture}
\newtheorem{corollary}[theorem]{Corollary}
\newtheorem{criterion}[theorem]{Criterion}
\newtheorem{definition}[theorem]{Definition}
\newtheorem{example}[theorem]{Example}
\newtheorem{exercise}[theorem]{Exercise}
\newtheorem{lemma}[theorem]{Lemma}
\newtheorem{notation}[theorem]{Notation}
\newtheorem{problem}[theorem]{Problem}
\newtheorem{proposition}[theorem]{Proposition}
\newtheorem{remark}[theorem]{Remark}
\newtheorem{solution}[theorem]{Solution}
\newtheorem{summary}[theorem]{Summary}
\newenvironment{proof}[1][Proof]{\noindent\textbf{#1.} }{\ \rule{0.5em}{0.5em}}
\input{tcilatex}

\begin{document}


\subsection{Model Assumptions}

The world economy is composed of $N$ countries. At a given point in time $t$%
, each country $n$ is endowed with $L_{nt}$ units of a primary (non
produced) input, which we interpret as equipped labour. There are $J$
sectors (or broad classes of goods) in the economy, whose output is combined
into a final good through a Cobb-Douglas aggregate. In formulas, aggregate
gross output in the economy is given by: 
\begin{equation}
Q_{nt}=\prod_{j=1}^{J}\left( Q_{nt}^{j}\right) ^{\alpha ^{j}}
\label{aggregate}
\end{equation}%
where $Q_{t}^{j}$ is the gross output in sector $j$ and $\sum_{j=1}^{J}%
\alpha ^{j}=1$. Competitive firms in each sector $j$ produce a composite
good according to the following constant-elasticity-of-substitution (CES)
technology: 
\begin{equation}
Q_{nt}^{j}=\left[ \int_{0}^{1}q_{nt}(\omega ^{j})^{\frac{\eta -1}{\eta }%
}d\omega ^{j}\right] ^{\frac{\eta }{\eta -1}}  \label{sectoroutput}
\end{equation}%
where $q_{nt}(\omega ^{j})$ is the quantity of good $\omega ^{j}$ used by
country $n$ in sector $j$ at time $t$, and $\eta >0$ is the elasticity of
substitution across goods within a given sector. The intermediate goods $%
\omega ^{j}$ can be produced locally or imported from other countries.
Delivering a good from country $n$ to country $m$ in sector $j$ and time
period $t$ results in $0<\kappa _{mnt}^{j}\leq 1$ goods arriving at $m$; we
assume that $\kappa _{mnt}^{j}\geq \kappa _{mkt}^{j}\kappa _{knt}^{j}\quad
\forall m,n,k,j,t$ and $\kappa _{nnt}^{j}=1$. All costs incurred are net
losses.\footnote{%
In the calibration, the $\kappa $s will reflect all trading costs, including
tariffs; so implicitly we adopt the extreme assumption that tariff revenues
are wasted---or at least not rebated back to agents in a way that would
interact with the allocation of resources in the economy.} Under the
assumption of perfect competition, goods are sourced from the lowest-cost
producer, after adjusting for transport costs. The technology for producing $%
q_{nt}(\omega ^{j})$ is given accordingly by the country of origin ($m$)
with the lowest cost (with $m=n$ when the good is produced locally): 
\begin{equation}
x_{mt}(\omega ^{j})=A_{mt}^{j}z_{m}(\omega ^{j})l_{mt}(\omega ^{j})^{\beta
^{j}}M_{mt}(\omega ^{j})^{1-\beta ^{j}}
\end{equation}%
where $x_{mt}(\omega ^{j})$ is the production of good $\omega ^{j}$ by
country $m$ at time $t$, $M_{mt}(\omega ^{j})$ is the amount of the
aggregate composite good used by country $m$ to produce $x_{mt}(\omega ^{j})$
units of good $\omega ^{j}$ and $l_{mt}(\omega ^{j})$ is the corresponding
amount of equipped labour. Total factor productivity (TFP) levels vary
across countries, sectors, and goods. Specifically, each intermediate good $%
\omega ^{j}$ in sector $j$ of country $n$ has a time-invariant idiosyncratic
productivity factor $z_{n}(\omega ^{j})$ and a time-varying factor $%
A_{nt}^{j}$ common to all the goods $\omega ^{j}$ in sector $j$. Building on
the literature, we assume the productivities $z_{n}(\omega ^{j})$ follow a
sector-specific, time-invariant Fr\'{e}chet distribution $F_{n}^{j}(z)=\exp
(-T_{n}^{j}z^{-\theta })$. A higher $T_{n}^{j}$ shifts the distribution of
productivities to the right, that is leading to probabilistically higher
productivities. A higher $\theta $ decreases the dispersion of the
productivity distribution, and hence reduces the scope for comparative
advantage. Shocks to $A_{nt}^{j}$ over time are interpreted as standard
sectoral total factor productivity (TFP) shocks.

The single final good can be used both as input in the production of
intermediaries $\omega^{j}$ or for final consumption, $C_{nt}$. Hence,
market clearing in the good markets implies: 
\[
Q_{nt} = C_{nt} + \sum_{j=1}^{J}\int_{0}^{1}M_{nt}(\omega^{j})d\omega ^{j},
\]
where the integral aggregates over the unit-size continuum of goods $%
\omega^{j}$ entering in the production of each sector's $j$ aggregate good.

Clearing in the input market within a sector implies: 
\[
L_{nt}^{j}=\int_{0}^{1}l_{nt}(\omega ^{j})d\omega ^{j},
\]%
where $l_{nt}(\omega ^{j})$ denotes the amount of equipped labour used in
the production of good $\omega ^{j}$ by country $n$. We assume there is
perfect risk-sharing within a country, but no risk-sharing across countries.%
\footnote{%
To motivate the lack of risk-sharing across countries, see our discussion of
Figures 1 and 2.} The (equipped) labour shares allocated to each sector, $%
L_{nt}^{j}/L_{nt},$ with $\sum\nolimits_{k=1}^{J}L_{nt}^{k}=L_{nt}$, are
determined ex ante (before the realization of the shocks), but can be
adjusted ex-post, at an adjustment cost. At the beginning of each period, a
representative consumer decides on the ex-ante optimal allocation of the
primary input $L_{nt}$ into different sectors in order to maximize the
expected value of utility. Then (stochastic) shocks to productivity $%
A_{nt}^{j}$ are realized; equipped labour is reallocated freely within a
sector, while reallocation across sectors implies a disutility cost $%
\Upsilon .$ Production and consumption take place. The disutility cost of
ex-post reallocation across sectors  aims at capturing the idea that in the
short run, it is costly to reallocate productive factors across sectors. 

The representative consumer's budget constraint in each period is: 
\[
P_{nt}C_{nt}=\sum_{j=1}^{J}w_{nt}^{j}L_{nt}^{j},
\]%
where $P_{nt}$ is the price of the aggregate good (\ref{aggregate}), $%
w_{nt}^{j}L_{nt}^{j}$ is the nominal value-added generated in sector $j$.
Lifetime utility is given by 
\[
U_{n}=\sum\limits_{t=0}^{\infty }\delta ^{t}u(C_{nt})+\Upsilon ,
\]%
where $u^{\prime }>0$, $u^{\prime \prime }\leq 0$ and $\delta $ is the
discount factor and $\Upsilon $ is the sum of quadratic deviations of
sectoral shares from the original sectoral allocation. The cost $\Upsilon $
is relevant ex-post, when consumers need to reallocate inputs across
sectors. Ex ante, however, the optimal allocation is such  that the expected
adjustment costs will be zero. Because there is no intertemporal trade and
no capital in the economy, and given that input shares ex ante will equal
the expected optimal allocation, each period consumers maximize the expected
static utility flow $E\left[ u\left( C_{nt}\right) \right] $ and the
equilibrium is simply a sequence of static equilibria. In making his labor
allocation decisions the representative consumer takes into account the
joint probability distribution function of sectoral productivities, $%
A_{nt}^{j}$s.

In the analysis, we assume log utility and therefore the consumer solves: 
\begin{equation}
L_{nt}^{k\ast }=\arg \max E_{t-1}\left[ \ln \left( \frac{%
\sum_{j=1}^{J}w_{nt}^{j}L_{nt}^{j}}{P_{nt}}\right) \right]
,s.t.:\sum\nolimits_{j=1}^{J}L_{nt}^{j}=L_{nt},  \label{eq:log:utility}
\end{equation}%
where $E_{t-1}$ indicates that the expectation is taken before the
realization of period $t$ shocks. This\ maximization problem leads to the
following first-order conditions for the ex-ante allocation of inputs to
sectors: 
\begin{equation}
\phi _{nt}^{k\ast }=\frac{L_{nt}^{k\ast }}{L_{nt}}=E_{t-1}\left[ \frac{%
w_{nt}^{k}L_{nt}^{k}}{\sum_{k}w_{nt}^{k}L_{nt}^{k}}\right] ,\qquad \forall
k,t.  \label{eq:FOC}
\end{equation}%
In words, the share of resources allocated to a given sector equals its
expected share in value added. To gain intuition on this expression note
that $1/\sum_{k}w_{nt}^{k}L_{nt}^{k}$ is the marginal utility of consumption
in period $t$; thus, more resources are allocated to higher value-added
sectors, after appropriately weighting by marginal utility. Consider, for
further intuition, a (small) sector whose productivity is negatively
correlated with the rest of the economy (that is, it has high value added
when the rest of the economy has low value added); in states of the world in
which overall income is low, the marginal utility of consumption $%
1/\sum_{k}w_{nt}^{k}L_{nt}^{k}$ will be high and hence the optimal
allocation entails allocating more resources to this sector.
(Log-linearizing this expression makes the role of second moments on the
allocation of resources clearer.) In the closed economy, the value-added
share is pinned down by the Cobb--Douglas coefficients $\alpha ^{j}\beta ^{j}
$, as with Cobb-Douglas technology there is no variation on expenditures
(and sales) shares---and log-utility implies the shares determine the
sectoral allocation of resources. (In the open economy this result no longer
holds as a country's sectoral shares depend on its absolute and comparative
advantage as well as trading costs vis-\`{a}-vis other countries.

Ex post, as said, labour can potentially be reallocated after incurring the
ajustment costs. The ex-post input shares solve:

\begin{equation}
L_{nt}^{k}=\arg \max \left[ \ln \left( \frac{%
\sum_{j=1}^{J}w_{nt}^{j}L_{nt}^{j}}{P_{nt}}\right) +\frac{\varrho }{2}%
\sum_{j=1}^{J}\left[ \phi _{nt}^{j}-\phi _{nt}^{j\ast }\right] ^{2}\right]
,s.t.:\sum\nolimits_{j=1}^{J}L_{nt}^{j}=L_{nt},\text{ }\phi _{nt}^{j}=\frac{%
L_{nt}^{j}}{L_{nt}}
\end{equation}

with $\phi _{nt}^{j}=\frac{L_{nt}^{j}}{L_{nt}}$ and $\sum_{j=1}^{J}\phi
_{nt}^{j}=\sum_{j=1}^{J}\phi _{nt}^{j\ast }=1.$ The first order conditions
satisfy%
\begin{equation}
\frac{w_{nt}^{k}}{\sum_{j=1}^{J}w_{nt}^{j}L_{nt}^{k}}+\varrho \left[ \phi
_{nt}^{k}-\phi _{nt}^{k\ast }\right] =\lambda 
\end{equation}%
where $\lambda $ is given by $\lambda =\frac{\frac{1}{J}%
\sum_{j=1}^{J}w_{nt}^{j}}{\sum_{j=1}^{J}w_{nt}^{j}L_{nt}^{j}}$ and hence%
\[
\phi _{nt}^{k}=\phi _{nt}^{k\ast }+\frac{1}{\varrho }\left[ \frac{\frac{1}{J}%
\sum_{j=1}^{J}w_{nt}^{j}-w_{nt}^{k}}{\sum_{j=1}^{J}w_{nt}^{j}L_{nt}^{j}}%
\right] 
\]%
In words, the ex pot input shares $\phi _{nt}^{k}$ equal the ex-ante optimal
shares $\phi _{nt}^{k\ast }$ plus the differential between the average
equipped labour cost in the economy $\frac{1}{J}\sum_{j=1}^{J}w_{nt}^{j}$
and the sectoral input cost $w_{nt}^{k}$. The adjustment cost parameter $%
\varrho $ determines the elasticity of adjustment. In the extreme, when
adjustment costs are infinite, the economy simply stays with the initial
sectoral shares $\phi _{nt}^{k}=\phi _{nt}^{k\ast }$. As adjustment costs
decrease, the sectoral shares adjust with the cost differential.

\bigskip 

READ UP TO HEAR!

\end{document}
