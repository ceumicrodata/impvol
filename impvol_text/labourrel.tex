%2multibyte Version: 5.50.0.2960 CodePage: 65001

\documentclass{article}
%%%%%%%%%%%%%%%%%%%%%%%%%%%%%%%%%%%%%%%%%%%%%%%%%%%%%%%%%%%%%%%%%%%%%%%%%%%%%%%%%%%%%%%%%%%%%%%%%%%%%%%%%%%%%%%%%%%%%%%%%%%%%%%%%%%%%%%%%%%%%%%%%%%%%%%%%%%%%%%%%%%%%%%%%%%%%%%%%%%%%%%%%%%%%%%%%%%%%%%%%%%%%%%%%%%%%%%%%%%%%%%%%%%%%%%%%%%%%%%%%%%%%%%%%%%%
\usepackage{amsmath}

\setcounter{MaxMatrixCols}{10}
%TCIDATA{OutputFilter=LATEX.DLL}
%TCIDATA{Version=5.50.0.2960}
%TCIDATA{Codepage=65001}
%TCIDATA{<META NAME="SaveForMode" CONTENT="1">}
%TCIDATA{BibliographyScheme=Manual}
%TCIDATA{Created=Monday, November 30, 2015 16:55:25}
%TCIDATA{LastRevised=Wednesday, December 09, 2015 17:26:48}
%TCIDATA{<META NAME="GraphicsSave" CONTENT="32">}
%TCIDATA{<META NAME="DocumentShell" CONTENT="Standard LaTeX\Blank - Standard LaTeX Article">}
%TCIDATA{CSTFile=40 LaTeX article.cst}

\newtheorem{theorem}{Theorem}
\newtheorem{acknowledgement}[theorem]{Acknowledgement}
\newtheorem{algorithm}[theorem]{Algorithm}
\newtheorem{axiom}[theorem]{Axiom}
\newtheorem{case}[theorem]{Case}
\newtheorem{claim}[theorem]{Claim}
\newtheorem{conclusion}[theorem]{Conclusion}
\newtheorem{condition}[theorem]{Condition}
\newtheorem{conjecture}[theorem]{Conjecture}
\newtheorem{corollary}[theorem]{Corollary}
\newtheorem{criterion}[theorem]{Criterion}
\newtheorem{definition}[theorem]{Definition}
\newtheorem{example}[theorem]{Example}
\newtheorem{exercise}[theorem]{Exercise}
\newtheorem{lemma}[theorem]{Lemma}
\newtheorem{notation}[theorem]{Notation}
\newtheorem{problem}[theorem]{Problem}
\newtheorem{proposition}[theorem]{Proposition}
\newtheorem{remark}[theorem]{Remark}
\newtheorem{solution}[theorem]{Solution}
\newtheorem{summary}[theorem]{Summary}
\newenvironment{proof}[1][Proof]{\noindent\textbf{#1.} }{\ \rule{0.5em}{0.5em}}
% Macros for Scientific Word 2.5 documents saved with the LaTeX filter.
%Copyright (C) 1994-95 TCI Software Research, Inc.
\typeout{TCILATEX Macros for Scientific Word 2.5 <22 Dec 95>.}
\typeout{NOTICE:  This macro file is NOT proprietary and may be 
freely copied and distributed.}
%
\makeatletter
%
%%%%%%%%%%%%%%%%%%%%%%
% macros for time
\newcount\@hour\newcount\@minute\chardef\@x10\chardef\@xv60
\def\tcitime{
\def\@time{%
  \@minute\time\@hour\@minute\divide\@hour\@xv
  \ifnum\@hour<\@x 0\fi\the\@hour:%
  \multiply\@hour\@xv\advance\@minute-\@hour
  \ifnum\@minute<\@x 0\fi\the\@minute
  }}%

%%%%%%%%%%%%%%%%%%%%%%
% macro for hyperref
\@ifundefined{hyperref}{\def\hyperref#1#2#3#4{#2\ref{#4}#3}}{}

% macro for external program call
\@ifundefined{qExtProgCall}{\def\qExtProgCall#1#2#3#4#5#6{\relax}}{}
%%%%%%%%%%%%%%%%%%%%%%
%
% macros for graphics
%
\def\FILENAME#1{#1}%
%
\def\QCTOpt[#1]#2{%
  \def\QCTOptB{#1}
  \def\QCTOptA{#2}
}
\def\QCTNOpt#1{%
  \def\QCTOptA{#1}
  \let\QCTOptB\empty
}
\def\Qct{%
  \@ifnextchar[{%
    \QCTOpt}{\QCTNOpt}
}
\def\QCBOpt[#1]#2{%
  \def\QCBOptB{#1}
  \def\QCBOptA{#2}
}
\def\QCBNOpt#1{%
  \def\QCBOptA{#1}
  \let\QCBOptB\empty
}
\def\Qcb{%
  \@ifnextchar[{%
    \QCBOpt}{\QCBNOpt}
}
\def\PrepCapArgs{%
  \ifx\QCBOptA\empty
    \ifx\QCTOptA\empty
      {}%
    \else
      \ifx\QCTOptB\empty
        {\QCTOptA}%
      \else
        [\QCTOptB]{\QCTOptA}%
      \fi
    \fi
  \else
    \ifx\QCBOptA\empty
      {}%
    \else
      \ifx\QCBOptB\empty
        {\QCBOptA}%
      \else
        [\QCBOptB]{\QCBOptA}%
      \fi
    \fi
  \fi
}
\newcount\GRAPHICSTYPE
%\GRAPHICSTYPE 0 is for TurboTeX
%\GRAPHICSTYPE 1 is for DVIWindo (PostScript)
%%%(removed)%\GRAPHICSTYPE 2 is for psfig (PostScript)
\GRAPHICSTYPE=\z@
\def\GRAPHICSPS#1{%
 \ifcase\GRAPHICSTYPE%\GRAPHICSTYPE=0
   \special{ps: #1}%
 \or%\GRAPHICSTYPE=1
   \special{language "PS", include "#1"}%
%%%\or%\GRAPHICSTYPE=2
%%%  #1%
 \fi
}%
%
\def\GRAPHICSHP#1{\special{include #1}}%
%
% \graffile{ body }                                  %#1
%          { contentswidth (scalar)  }               %#2
%          { contentsheight (scalar) }               %#3
%          { vertical shift when in-line (scalar) }  %#4
\def\graffile#1#2#3#4{%
%%% \ifnum\GRAPHICSTYPE=\tw@
%%%  %Following if using psfig
%%%  \@ifundefined{psfig}{\input psfig.tex}{}%
%%%  \psfig{file=#1, height=#3, width=#2}%
%%% \else
  %Following for all others
  % JCS - added BOXTHEFRAME, see below
    \leavevmode
    \raise -#4 \BOXTHEFRAME{%
        \hbox to #2{\raise #3\hbox to #2{\null #1\hfil}}}%
}%
%
% A box for drafts
\def\draftbox#1#2#3#4{%
 \leavevmode\raise -#4 \hbox{%
  \frame{\rlap{\protect\tiny #1}\hbox to #2%
   {\vrule height#3 width\z@ depth\z@\hfil}%
  }%
 }%
}%
%
\newcount\draft
\draft=\z@
\let\nographics=\draft
\newif\ifwasdraft
\wasdraftfalse

%  \GRAPHIC{ body }                                  %#1
%          { draft name }                            %#2
%          { contentswidth (scalar)  }               %#3
%          { contentsheight (scalar) }               %#4
%          { vertical shift when in-line (scalar) }  %#5
\def\GRAPHIC#1#2#3#4#5{%
 \ifnum\draft=\@ne\draftbox{#2}{#3}{#4}{#5}%
  \else\graffile{#1}{#3}{#4}{#5}%
  \fi
 }%
%
\def\addtoLaTeXparams#1{%
    \edef\LaTeXparams{\LaTeXparams #1}}%
%
% JCS -  added a switch BoxFrame that can 
% be set by including X in the frame params.
% If set a box is drawn around the frame.

\newif\ifBoxFrame \BoxFramefalse
\newif\ifOverFrame \OverFramefalse
\newif\ifUnderFrame \UnderFramefalse

\def\BOXTHEFRAME#1{%
   \hbox{%
      \ifBoxFrame
         \frame{#1}%
      \else
         {#1}%
      \fi
   }%
}


\def\doFRAMEparams#1{\BoxFramefalse\OverFramefalse\UnderFramefalse\readFRAMEparams#1\end}%
\def\readFRAMEparams#1{%
 \ifx#1\end%
  \let\next=\relax
  \else
  \ifx#1i\dispkind=\z@\fi
  \ifx#1d\dispkind=\@ne\fi
  \ifx#1f\dispkind=\tw@\fi
  \ifx#1t\addtoLaTeXparams{t}\fi
  \ifx#1b\addtoLaTeXparams{b}\fi
  \ifx#1p\addtoLaTeXparams{p}\fi
  \ifx#1h\addtoLaTeXparams{h}\fi
  \ifx#1X\BoxFrametrue\fi
  \ifx#1O\OverFrametrue\fi
  \ifx#1U\UnderFrametrue\fi
  \ifx#1w
    \ifnum\draft=1\wasdrafttrue\else\wasdraftfalse\fi
    \draft=\@ne
  \fi
  \let\next=\readFRAMEparams
  \fi
 \next
 }%
%
%Macro for In-line graphics object
%   \IFRAME{ contentswidth (scalar)  }               %#1
%          { contentsheight (scalar) }               %#2
%          { vertical shift when in-line (scalar) }  %#3
%          { draft name }                            %#4
%          { body }                                  %#5
%          { caption}                                %#6


\def\IFRAME#1#2#3#4#5#6{%
      \bgroup
      \let\QCTOptA\empty
      \let\QCTOptB\empty
      \let\QCBOptA\empty
      \let\QCBOptB\empty
      #6%
      \parindent=0pt%
      \leftskip=0pt
      \rightskip=0pt
      \setbox0 = \hbox{\QCBOptA}%
      \@tempdima = #1\relax
      \ifOverFrame
          % Do this later
          \typeout{This is not implemented yet}%
          \show\HELP
      \else
         \ifdim\wd0>\@tempdima
            \advance\@tempdima by \@tempdima
            \ifdim\wd0 >\@tempdima
               \textwidth=\@tempdima
               \setbox1 =\vbox{%
                  \noindent\hbox to \@tempdima{\hfill\GRAPHIC{#5}{#4}{#1}{#2}{#3}\hfill}\\%
                  \noindent\hbox to \@tempdima{\parbox[b]{\@tempdima}{\QCBOptA}}%
               }%
               \wd1=\@tempdima
            \else
               \textwidth=\wd0
               \setbox1 =\vbox{%
                 \noindent\hbox to \wd0{\hfill\GRAPHIC{#5}{#4}{#1}{#2}{#3}\hfill}\\%
                 \noindent\hbox{\QCBOptA}%
               }%
               \wd1=\wd0
            \fi
         \else
            %\show\BBB
            \ifdim\wd0>0pt
              \hsize=\@tempdima
              \setbox1 =\vbox{%
                \unskip\GRAPHIC{#5}{#4}{#1}{#2}{0pt}%
                \break
                \unskip\hbox to \@tempdima{\hfill \QCBOptA\hfill}%
              }%
              \wd1=\@tempdima
           \else
              \hsize=\@tempdima
              \setbox1 =\vbox{%
                \unskip\GRAPHIC{#5}{#4}{#1}{#2}{0pt}%
              }%
              \wd1=\@tempdima
           \fi
         \fi
         \@tempdimb=\ht1
         \advance\@tempdimb by \dp1
         \advance\@tempdimb by -#2%
         \advance\@tempdimb by #3%
         \leavevmode
         \raise -\@tempdimb \hbox{\box1}%
      \fi
      \egroup%
}%
%
%Macro for Display graphics object
%   \DFRAME{ contentswidth (scalar)  }               %#1
%          { contentsheight (scalar) }               %#2
%          { draft label }                           %#3
%          { name }                                  %#4
%          { caption}                                %#5
\def\DFRAME#1#2#3#4#5{%
 \begin{center}
     \let\QCTOptA\empty
     \let\QCTOptB\empty
     \let\QCBOptA\empty
     \let\QCBOptB\empty
     \ifOverFrame 
        #5\QCTOptA\par
     \fi
     \GRAPHIC{#4}{#3}{#1}{#2}{\z@}
     \ifUnderFrame 
        \nobreak\par #5\QCBOptA
     \fi
 \end{center}%
 }%
%
%Macro for Floating graphic object
%   \FFRAME{ framedata f|i tbph x F|T }              %#1
%          { contentswidth (scalar)  }               %#2
%          { contentsheight (scalar) }               %#3
%          { caption }                               %#4
%          { label }                                 %#5
%          { draft name }                            %#6
%          { body }                                  %#7
\def\FFRAME#1#2#3#4#5#6#7{%
 \begin{figure}[#1]%
  \let\QCTOptA\empty
  \let\QCTOptB\empty
  \let\QCBOptA\empty
  \let\QCBOptB\empty
  \ifOverFrame
    #4
    \ifx\QCTOptA\empty
    \else
      \ifx\QCTOptB\empty
        \caption{\QCTOptA}%
      \else
        \caption[\QCTOptB]{\QCTOptA}%
      \fi
    \fi
    \ifUnderFrame\else
      \label{#5}%
    \fi
  \else
    \UnderFrametrue%
  \fi
  \begin{center}\GRAPHIC{#7}{#6}{#2}{#3}{\z@}\end{center}%
  \ifUnderFrame
    #4
    \ifx\QCBOptA\empty
      \caption{}%
    \else
      \ifx\QCBOptB\empty
        \caption{\QCBOptA}%
      \else
        \caption[\QCBOptB]{\QCBOptA}%
      \fi
    \fi
    \label{#5}%
  \fi
  \end{figure}%
 }%
%
%
%    \FRAME{ framedata f|i tbph x F|T }              %#1
%          { contentswidth (scalar)  }               %#2
%          { contentsheight (scalar) }               %#3
%          { vertical shift when in-line (scalar) }  %#4
%          { caption }                               %#5
%          { label }                                 %#6
%          { name }                                  %#7
%          { body }                                  %#8
%
%    framedata is a string which can contain the following
%    characters: idftbphxFT
%    Their meaning is as follows:
%             i, d or f : in-line, display, or floating
%             t,b,p,h   : LaTeX floating placement options
%             x         : fit contents box to contents
%             F or T    : Figure or Table. 
%                         Later this can expand
%                         to a more general float class.
%
%
\newcount\dispkind%

\def\makeactives{
  \catcode`\"=\active
  \catcode`\;=\active
  \catcode`\:=\active
  \catcode`\'=\active
  \catcode`\~=\active
}
\bgroup
   \makeactives
   \gdef\activesoff{%
      \def"{\string"}
      \def;{\string;}
      \def:{\string:}
      \def'{\string'}
      \def~{\string~}
      %\bbl@deactivate{"}%
      %\bbl@deactivate{;}%
      %\bbl@deactivate{:}%
      %\bbl@deactivate{'}%
    }
\egroup

\def\FRAME#1#2#3#4#5#6#7#8{%
 \bgroup
 \@ifundefined{bbl@deactivate}{}{\activesoff}
 \ifnum\draft=\@ne
   \wasdrafttrue
 \else
   \wasdraftfalse%
 \fi
 \def\LaTeXparams{}%
 \dispkind=\z@
 \def\LaTeXparams{}%
 \doFRAMEparams{#1}%
 \ifnum\dispkind=\z@\IFRAME{#2}{#3}{#4}{#7}{#8}{#5}\else
  \ifnum\dispkind=\@ne\DFRAME{#2}{#3}{#7}{#8}{#5}\else
   \ifnum\dispkind=\tw@
    \edef\@tempa{\noexpand\FFRAME{\LaTeXparams}}%
    \@tempa{#2}{#3}{#5}{#6}{#7}{#8}%
    \fi
   \fi
  \fi
  \ifwasdraft\draft=1\else\draft=0\fi{}%
  \egroup
 }%
%
% This macro added to let SW gobble a parameter that
% should not be passed on and expanded. 

\def\TEXUX#1{"texux"}

%
% Macros for text attributes:
%
\def\BF#1{{\bf {#1}}}%
\def\NEG#1{\leavevmode\hbox{\rlap{\thinspace/}{$#1$}}}%
%
%%%%%%%%%%%%%%%%%%%%%%%%%%%%%%%%%%%%%%%%%%%%%%%%%%%%%%%%%%%%%%%%%%%%%%%%
%
%
% macros for user - defined functions
\def\func#1{\mathop{\rm #1}}%
\def\limfunc#1{\mathop{\rm #1}}%

%
% miscellaneous 
%\long\def\QQQ#1#2{}%
\long\def\QQQ#1#2{%
     \long\expandafter\def\csname#1\endcsname{#2}}%
%\def\QTP#1{}% JCS - this was changed becuase style editor will define QTP
\@ifundefined{QTP}{\def\QTP#1{}}{}
\@ifundefined{QEXCLUDE}{\def\QEXCLUDE#1{}}{}
%\@ifundefined{Qcb}{\def\Qcb#1{#1}}{}
%\@ifundefined{Qct}{\def\Qct#1{#1}}{}
\@ifundefined{Qlb}{\def\Qlb#1{#1}}{}
\@ifundefined{Qlt}{\def\Qlt#1{#1}}{}
\def\QWE{}%
\long\def\QQA#1#2{}%
%\def\QTR#1#2{{\em #2}}% Always \em!!!
%\def\QTR#1#2{\mbox{\begin{#1}#2\end{#1}}}%cb%%%
\def\QTR#1#2{{\csname#1\endcsname #2}}%(gp) Is this the best?
\long\def\TeXButton#1#2{#2}%
\long\def\QSubDoc#1#2{#2}%
\def\EXPAND#1[#2]#3{}%
\def\NOEXPAND#1[#2]#3{}%
\def\PROTECTED{}%
\def\LaTeXparent#1{}%
\def\ChildStyles#1{}%
\def\ChildDefaults#1{}%
\def\QTagDef#1#2#3{}%
%
% Macros for style editor docs
\@ifundefined{StyleEditBeginDoc}{\def\StyleEditBeginDoc{\relax}}{}
%
% Macros for footnotes
\def\QQfnmark#1{\footnotemark}
\def\QQfntext#1#2{\addtocounter{footnote}{#1}\footnotetext{#2}}
%
% Macros for indexing.
\def\MAKEINDEX{\makeatletter\input gnuindex.sty\makeatother\makeindex}%	
\@ifundefined{INDEX}{\def\INDEX#1#2{}{}}{}%
\@ifundefined{SUBINDEX}{\def\SUBINDEX#1#2#3{}{}{}}{}%
\@ifundefined{initial}%  
   {\def\initial#1{\bigbreak{\raggedright\large\bf #1}\kern 2\p@\penalty3000}}%
   {}%
\@ifundefined{entry}{\def\entry#1#2{\item {#1}, #2}}{}%
\@ifundefined{primary}{\def\primary#1{\item {#1}}}{}%
\@ifundefined{secondary}{\def\secondary#1#2{\subitem {#1}, #2}}{}%
%
%
\@ifundefined{ZZZ}{}{\MAKEINDEX\makeatletter}%
%
% Attempts to avoid problems with other styles
\@ifundefined{abstract}{%
 \def\abstract{%
  \if@twocolumn
   \section*{Abstract (Not appropriate in this style!)}%
   \else \small 
   \begin{center}{\bf Abstract\vspace{-.5em}\vspace{\z@}}\end{center}%
   \quotation 
   \fi
  }%
 }{%
 }%
\@ifundefined{endabstract}{\def\endabstract
  {\if@twocolumn\else\endquotation\fi}}{}%
\@ifundefined{maketitle}{\def\maketitle#1{}}{}%
\@ifundefined{affiliation}{\def\affiliation#1{}}{}%
\@ifundefined{proof}{\def\proof{\noindent{\bfseries Proof. }}}{}%
\@ifundefined{endproof}{\def\endproof{\mbox{\ \rule{.1in}{.1in}}}}{}%
\@ifundefined{newfield}{\def\newfield#1#2{}}{}%
\@ifundefined{chapter}{\def\chapter#1{\par(Chapter head:)#1\par }%
 \newcount\c@chapter}{}%
\@ifundefined{part}{\def\part#1{\par(Part head:)#1\par }}{}%
\@ifundefined{section}{\def\section#1{\par(Section head:)#1\par }}{}%
\@ifundefined{subsection}{\def\subsection#1%
 {\par(Subsection head:)#1\par }}{}%
\@ifundefined{subsubsection}{\def\subsubsection#1%
 {\par(Subsubsection head:)#1\par }}{}%
\@ifundefined{paragraph}{\def\paragraph#1%
 {\par(Subsubsubsection head:)#1\par }}{}%
\@ifundefined{subparagraph}{\def\subparagraph#1%
 {\par(Subsubsubsubsection head:)#1\par }}{}%
%%%%%%%%%%%%%%%%%%%%%%%%%%%%%%%%%%%%%%%%%%%%%%%%%%%%%%%%%%%%%%%%%%%%%%%%
% These symbols are not recognized by LaTeX
\@ifundefined{therefore}{\def\therefore{}}{}%
\@ifundefined{backepsilon}{\def\backepsilon{}}{}%
\@ifundefined{yen}{\def\yen{\hbox{\rm\rlap=Y}}}{}%
\@ifundefined{registered}{%
   \def\registered{\relax\ifmmode{}\r@gistered
                    \else$\m@th\r@gistered$\fi}%
 \def\r@gistered{^{\ooalign
  {\hfil\raise.07ex\hbox{$\scriptstyle\rm\text{R}$}\hfil\crcr
  \mathhexbox20D}}}}{}%
\@ifundefined{Eth}{\def\Eth{}}{}%
\@ifundefined{eth}{\def\eth{}}{}%
\@ifundefined{Thorn}{\def\Thorn{}}{}%
\@ifundefined{thorn}{\def\thorn{}}{}%
% A macro to allow any symbol that requires math to appear in text
\def\TEXTsymbol#1{\mbox{$#1$}}%
\@ifundefined{degree}{\def\degree{{}^{\circ}}}{}%
%
% macros for T3TeX files
\newdimen\theight
\def\Column{%
 \vadjust{\setbox\z@=\hbox{\scriptsize\quad\quad tcol}%
  \theight=\ht\z@\advance\theight by \dp\z@\advance\theight by \lineskip
  \kern -\theight \vbox to \theight{%
   \rightline{\rlap{\box\z@}}%
   \vss
   }%
  }%
 }%
%
\def\qed{%
 \ifhmode\unskip\nobreak\fi\ifmmode\ifinner\else\hskip5\p@\fi\fi
 \hbox{\hskip5\p@\vrule width4\p@ height6\p@ depth1.5\p@\hskip\p@}%
 }%
%
\def\cents{\hbox{\rm\rlap/c}}%
\def\miss{\hbox{\vrule height2\p@ width 2\p@ depth\z@}}%
%\def\miss{\hbox{.}}%        %another possibility 
%
\def\vvert{\Vert}%           %always translated to \left| or \right|
%
\def\tcol#1{{\baselineskip=6\p@ \vcenter{#1}} \Column}  %
%
\def\dB{\hbox{{}}}%                 %dummy entry in column 
\def\mB#1{\hbox{$#1$}}%             %column entry
\def\nB#1{\hbox{#1}}%               %column entry (not math)
%
%\newcount\notenumber
%\def\clearnotenumber{\notenumber=0}
%\def\note{\global\advance\notenumber by 1
% \footnote{$^{\the\notenumber}$}}
%\def\note{\global\advance\notenumber by 1
\def\note{$^{\dag}}%
%
%

\def\newfmtname{LaTeX2e}
\def\chkcompat{%
   \if@compatibility
   \else
     \usepackage{latexsym}
   \fi
}

\ifx\fmtname\newfmtname
  \DeclareOldFontCommand{\rm}{\normalfont\rmfamily}{\mathrm}
  \DeclareOldFontCommand{\sf}{\normalfont\sffamily}{\mathsf}
  \DeclareOldFontCommand{\tt}{\normalfont\ttfamily}{\mathtt}
  \DeclareOldFontCommand{\bf}{\normalfont\bfseries}{\mathbf}
  \DeclareOldFontCommand{\it}{\normalfont\itshape}{\mathit}
  \DeclareOldFontCommand{\sl}{\normalfont\slshape}{\@nomath\sl}
  \DeclareOldFontCommand{\sc}{\normalfont\scshape}{\@nomath\sc}
  \chkcompat
\fi

%
% Greek bold macros
% Redefine all of the math symbols 
% which might be bolded	 - there are 
% probably others to add to this list

\def\alpha{\Greekmath 010B }%
\def\beta{\Greekmath 010C }%
\def\gamma{\Greekmath 010D }%
\def\delta{\Greekmath 010E }%
\def\epsilon{\Greekmath 010F }%
\def\zeta{\Greekmath 0110 }%
\def\eta{\Greekmath 0111 }%
\def\theta{\Greekmath 0112 }%
\def\iota{\Greekmath 0113 }%
\def\kappa{\Greekmath 0114 }%
\def\lambda{\Greekmath 0115 }%
\def\mu{\Greekmath 0116 }%
\def\nu{\Greekmath 0117 }%
\def\xi{\Greekmath 0118 }%
\def\pi{\Greekmath 0119 }%
\def\rho{\Greekmath 011A }%
\def\sigma{\Greekmath 011B }%
\def\tau{\Greekmath 011C }%
\def\upsilon{\Greekmath 011D }%
\def\phi{\Greekmath 011E }%
\def\chi{\Greekmath 011F }%
\def\psi{\Greekmath 0120 }%
\def\omega{\Greekmath 0121 }%
\def\varepsilon{\Greekmath 0122 }%
\def\vartheta{\Greekmath 0123 }%
\def\varpi{\Greekmath 0124 }%
\def\varrho{\Greekmath 0125 }%
\def\varsigma{\Greekmath 0126 }%
\def\varphi{\Greekmath 0127 }%

\def\nabla{\Greekmath 0272 }
\def\FindBoldGroup{%
   {\setbox0=\hbox{$\mathbf{x\global\edef\theboldgroup{\the\mathgroup}}$}}%
}

\def\Greekmath#1#2#3#4{%
    \if@compatibility
        \ifnum\mathgroup=\symbold
           \mathchoice{\mbox{\boldmath$\displaystyle\mathchar"#1#2#3#4$}}%
                      {\mbox{\boldmath$\textstyle\mathchar"#1#2#3#4$}}%
                      {\mbox{\boldmath$\scriptstyle\mathchar"#1#2#3#4$}}%
                      {\mbox{\boldmath$\scriptscriptstyle\mathchar"#1#2#3#4$}}%
        \else
           \mathchar"#1#2#3#4% 
        \fi 
    \else 
        \FindBoldGroup
        \ifnum\mathgroup=\theboldgroup % For 2e
           \mathchoice{\mbox{\boldmath$\displaystyle\mathchar"#1#2#3#4$}}%
                      {\mbox{\boldmath$\textstyle\mathchar"#1#2#3#4$}}%
                      {\mbox{\boldmath$\scriptstyle\mathchar"#1#2#3#4$}}%
                      {\mbox{\boldmath$\scriptscriptstyle\mathchar"#1#2#3#4$}}%
        \else
           \mathchar"#1#2#3#4% 
        \fi     	    
	  \fi}

\newif\ifGreekBold  \GreekBoldfalse
\let\SAVEPBF=\pbf
\def\pbf{\GreekBoldtrue\SAVEPBF}%
%

\@ifundefined{theorem}{\newtheorem{theorem}{Theorem}}{}
\@ifundefined{lemma}{\newtheorem{lemma}[theorem]{Lemma}}{}
\@ifundefined{corollary}{\newtheorem{corollary}[theorem]{Corollary}}{}
\@ifundefined{conjecture}{\newtheorem{conjecture}[theorem]{Conjecture}}{}
\@ifundefined{proposition}{\newtheorem{proposition}[theorem]{Proposition}}{}
\@ifundefined{axiom}{\newtheorem{axiom}{Axiom}}{}
\@ifundefined{remark}{\newtheorem{remark}{Remark}}{}
\@ifundefined{example}{\newtheorem{example}{Example}}{}
\@ifundefined{exercise}{\newtheorem{exercise}{Exercise}}{}
\@ifundefined{definition}{\newtheorem{definition}{Definition}}{}


\@ifundefined{mathletters}{%
  %\def\theequation{\arabic{equation}}
  \newcounter{equationnumber}  
  \def\mathletters{%
     \addtocounter{equation}{1}
     \edef\@currentlabel{\theequation}%
     \setcounter{equationnumber}{\c@equation}
     \setcounter{equation}{0}%
     \edef\theequation{\@currentlabel\noexpand\alph{equation}}%
  }
  \def\endmathletters{%
     \setcounter{equation}{\value{equationnumber}}%
  }
}{}

%Logos
\@ifundefined{BibTeX}{%
    \def\BibTeX{{\rm B\kern-.05em{\sc i\kern-.025em b}\kern-.08em
                 T\kern-.1667em\lower.7ex\hbox{E}\kern-.125emX}}}{}%
\@ifundefined{AmS}%
    {\def\AmS{{\protect\usefont{OMS}{cmsy}{m}{n}%
                A\kern-.1667em\lower.5ex\hbox{M}\kern-.125emS}}}{}%
\@ifundefined{AmSTeX}{\def\AmSTeX{\protect\AmS-\protect\TeX\@}}{}%
%

%%%%%%%%%%%%%%%%%%%%%%%%%%%%%%%%%%%%%%%%%%%%%%%%%%%%%%%%%%%%%%%%%%%%%%%
% NOTE: The rest of this file is read only if amstex has not been
% loaded.  This section is used to define amstex constructs in the
% event they have not been defined.
%
%
\ifx\ds@amstex\relax
   \message{amstex already loaded}\makeatother\endinput% 2.09 compatability
\else
   \@ifpackageloaded{amstex}%
      {\message{amstex already loaded}\makeatother\endinput}
      {}
   \@ifpackageloaded{amsgen}%
      {\message{amsgen already loaded}\makeatother\endinput}
      {}
\fi
%%%%%%%%%%%%%%%%%%%%%%%%%%%%%%%%%%%%%%%%%%%%%%%%%%%%%%%%%%%%%%%%%%%%%%%%
%%
%
%
%  Macros to define some AMS LaTeX constructs when 
%  AMS LaTeX has not been loaded
% 
% These macros are copied from the AMS-TeX package for doing
% multiple integrals.
%
\let\DOTSI\relax
\def\RIfM@{\relax\ifmmode}%
\def\FN@{\futurelet\next}%
\newcount\intno@
\def\iint{\DOTSI\intno@\tw@\FN@\ints@}%
\def\iiint{\DOTSI\intno@\thr@@\FN@\ints@}%
\def\iiiint{\DOTSI\intno@4 \FN@\ints@}%
\def\idotsint{\DOTSI\intno@\z@\FN@\ints@}%
\def\ints@{\findlimits@\ints@@}%
\newif\iflimtoken@
\newif\iflimits@
\def\findlimits@{\limtoken@true\ifx\next\limits\limits@true
 \else\ifx\next\nolimits\limits@false\else
 \limtoken@false\ifx\ilimits@\nolimits\limits@false\else
 \ifinner\limits@false\else\limits@true\fi\fi\fi\fi}%
\def\multint@{\int\ifnum\intno@=\z@\intdots@                          %1
 \else\intkern@\fi                                                    %2
 \ifnum\intno@>\tw@\int\intkern@\fi                                   %3
 \ifnum\intno@>\thr@@\int\intkern@\fi                                 %4
 \int}%                                                               %5
\def\multintlimits@{\intop\ifnum\intno@=\z@\intdots@\else\intkern@\fi
 \ifnum\intno@>\tw@\intop\intkern@\fi
 \ifnum\intno@>\thr@@\intop\intkern@\fi\intop}%
\def\intic@{%
    \mathchoice{\hskip.5em}{\hskip.4em}{\hskip.4em}{\hskip.4em}}%
\def\negintic@{\mathchoice
 {\hskip-.5em}{\hskip-.4em}{\hskip-.4em}{\hskip-.4em}}%
\def\ints@@{\iflimtoken@                                              %1
 \def\ints@@@{\iflimits@\negintic@
   \mathop{\intic@\multintlimits@}\limits                             %2
  \else\multint@\nolimits\fi                                          %3
  \eat@}%                                                             %4
 \else                                                                %5
 \def\ints@@@{\iflimits@\negintic@
  \mathop{\intic@\multintlimits@}\limits\else
  \multint@\nolimits\fi}\fi\ints@@@}%
\def\intkern@{\mathchoice{\!\!\!}{\!\!}{\!\!}{\!\!}}%
\def\plaincdots@{\mathinner{\cdotp\cdotp\cdotp}}%
\def\intdots@{\mathchoice{\plaincdots@}%
 {{\cdotp}\mkern1.5mu{\cdotp}\mkern1.5mu{\cdotp}}%
 {{\cdotp}\mkern1mu{\cdotp}\mkern1mu{\cdotp}}%
 {{\cdotp}\mkern1mu{\cdotp}\mkern1mu{\cdotp}}}%
%
%
%  These macros are for doing the AMS \text{} construct
%
\def\RIfM@{\relax\protect\ifmmode}
\def\text{\RIfM@\expandafter\text@\else\expandafter\mbox\fi}
\let\nfss@text\text
\def\text@#1{\mathchoice
   {\textdef@\displaystyle\f@size{#1}}%
   {\textdef@\textstyle\tf@size{\firstchoice@false #1}}%
   {\textdef@\textstyle\sf@size{\firstchoice@false #1}}%
   {\textdef@\textstyle \ssf@size{\firstchoice@false #1}}%
   \glb@settings}

\def\textdef@#1#2#3{\hbox{{%
                    \everymath{#1}%
                    \let\f@size#2\selectfont
                    #3}}}
\newif\iffirstchoice@
\firstchoice@true
%
%    Old Scheme for \text
%
%\def\rmfam{\z@}%
%\newif\iffirstchoice@
%\firstchoice@true
%\def\textfonti{\the\textfont\@ne}%
%\def\textfontii{\the\textfont\tw@}%
%\def\text{\RIfM@\expandafter\text@\else\expandafter\text@@\fi}%
%\def\text@@#1{\leavevmode\hbox{#1}}%
%\def\text@#1{\mathchoice
% {\hbox{\everymath{\displaystyle}\def\textfonti{\the\textfont\@ne}%
%  \def\textfontii{\the\textfont\tw@}\textdef@@ T#1}}%
% {\hbox{\firstchoice@false
%  \everymath{\textstyle}\def\textfonti{\the\textfont\@ne}%
%  \def\textfontii{\the\textfont\tw@}\textdef@@ T#1}}%
% {\hbox{\firstchoice@false
%  \everymath{\scriptstyle}\def\textfonti{\the\scriptfont\@ne}%
%  \def\textfontii{\the\scriptfont\tw@}\textdef@@ S\rm#1}}%
% {\hbox{\firstchoice@false
%  \everymath{\scriptscriptstyle}\def\textfonti
%  {\the\scriptscriptfont\@ne}%
%  \def\textfontii{\the\scriptscriptfont\tw@}\textdef@@ s\rm#1}}}%
%\def\textdef@@#1{\textdef@#1\rm\textdef@#1\bf\textdef@#1\sl
%    \textdef@#1\it}%
%\def\DN@{\def\next@}%
%\def\eat@#1{}%
%\def\textdef@#1#2{%
% \DN@{\csname\expandafter\eat@\string#2fam\endcsname}%
% \if S#1\edef#2{\the\scriptfont\next@\relax}%
% \else\if s#1\edef#2{\the\scriptscriptfont\next@\relax}%
% \else\edef#2{\the\textfont\next@\relax}\fi\fi}%
%
%
%These are the AMS constructs for multiline limits.
%
\def\Let@{\relax\iffalse{\fi\let\\=\cr\iffalse}\fi}%
\def\vspace@{\def\vspace##1{\crcr\noalign{\vskip##1\relax}}}%
\def\multilimits@{\bgroup\vspace@\Let@
 \baselineskip\fontdimen10 \scriptfont\tw@
 \advance\baselineskip\fontdimen12 \scriptfont\tw@
 \lineskip\thr@@\fontdimen8 \scriptfont\thr@@
 \lineskiplimit\lineskip
 \vbox\bgroup\ialign\bgroup\hfil$\m@th\scriptstyle{##}$\hfil\crcr}%
\def\Sb{_\multilimits@}%
\def\endSb{\crcr\egroup\egroup\egroup}%
\def\Sp{^\multilimits@}%
\let\endSp\endSb
%
%
%These are AMS constructs for horizontal arrows
%
\newdimen\ex@
\ex@.2326ex
\def\rightarrowfill@#1{$#1\m@th\mathord-\mkern-6mu\cleaders
 \hbox{$#1\mkern-2mu\mathord-\mkern-2mu$}\hfill
 \mkern-6mu\mathord\rightarrow$}%
\def\leftarrowfill@#1{$#1\m@th\mathord\leftarrow\mkern-6mu\cleaders
 \hbox{$#1\mkern-2mu\mathord-\mkern-2mu$}\hfill\mkern-6mu\mathord-$}%
\def\leftrightarrowfill@#1{$#1\m@th\mathord\leftarrow
\mkern-6mu\cleaders
 \hbox{$#1\mkern-2mu\mathord-\mkern-2mu$}\hfill
 \mkern-6mu\mathord\rightarrow$}%
\def\overrightarrow{\mathpalette\overrightarrow@}%
\def\overrightarrow@#1#2{\vbox{\ialign{##\crcr\rightarrowfill@#1\crcr
 \noalign{\kern-\ex@\nointerlineskip}$\m@th\hfil#1#2\hfil$\crcr}}}%
\let\overarrow\overrightarrow
\def\overleftarrow{\mathpalette\overleftarrow@}%
\def\overleftarrow@#1#2{\vbox{\ialign{##\crcr\leftarrowfill@#1\crcr
 \noalign{\kern-\ex@\nointerlineskip}$\m@th\hfil#1#2\hfil$\crcr}}}%
\def\overleftrightarrow{\mathpalette\overleftrightarrow@}%
\def\overleftrightarrow@#1#2{\vbox{\ialign{##\crcr
   \leftrightarrowfill@#1\crcr
 \noalign{\kern-\ex@\nointerlineskip}$\m@th\hfil#1#2\hfil$\crcr}}}%
\def\underrightarrow{\mathpalette\underrightarrow@}%
\def\underrightarrow@#1#2{\vtop{\ialign{##\crcr$\m@th\hfil#1#2\hfil
  $\crcr\noalign{\nointerlineskip}\rightarrowfill@#1\crcr}}}%
\let\underarrow\underrightarrow
\def\underleftarrow{\mathpalette\underleftarrow@}%
\def\underleftarrow@#1#2{\vtop{\ialign{##\crcr$\m@th\hfil#1#2\hfil
  $\crcr\noalign{\nointerlineskip}\leftarrowfill@#1\crcr}}}%
\def\underleftrightarrow{\mathpalette\underleftrightarrow@}%
\def\underleftrightarrow@#1#2{\vtop{\ialign{##\crcr$\m@th
  \hfil#1#2\hfil$\crcr
 \noalign{\nointerlineskip}\leftrightarrowfill@#1\crcr}}}%
%%%%%%%%%%%%%%%%%%%%%

% 94.0815 by Jon:

\def\qopnamewl@#1{\mathop{\operator@font#1}\nlimits@}
\let\nlimits@\displaylimits
\def\setboxz@h{\setbox\z@\hbox}


\def\varlim@#1#2{\mathop{\vtop{\ialign{##\crcr
 \hfil$#1\m@th\operator@font lim$\hfil\crcr
 \noalign{\nointerlineskip}#2#1\crcr
 \noalign{\nointerlineskip\kern-\ex@}\crcr}}}}

 \def\rightarrowfill@#1{\m@th\setboxz@h{$#1-$}\ht\z@\z@
  $#1\copy\z@\mkern-6mu\cleaders
  \hbox{$#1\mkern-2mu\box\z@\mkern-2mu$}\hfill
  \mkern-6mu\mathord\rightarrow$}
\def\leftarrowfill@#1{\m@th\setboxz@h{$#1-$}\ht\z@\z@
  $#1\mathord\leftarrow\mkern-6mu\cleaders
  \hbox{$#1\mkern-2mu\copy\z@\mkern-2mu$}\hfill
  \mkern-6mu\box\z@$}


\def\projlim{\qopnamewl@{proj\,lim}}
\def\injlim{\qopnamewl@{inj\,lim}}
\def\varinjlim{\mathpalette\varlim@\rightarrowfill@}
\def\varprojlim{\mathpalette\varlim@\leftarrowfill@}
\def\varliminf{\mathpalette\varliminf@{}}
\def\varliminf@#1{\mathop{\underline{\vrule\@depth.2\ex@\@width\z@
   \hbox{$#1\m@th\operator@font lim$}}}}
\def\varlimsup{\mathpalette\varlimsup@{}}
\def\varlimsup@#1{\mathop{\overline
  {\hbox{$#1\m@th\operator@font lim$}}}}

%
%%%%%%%%%%%%%%%%%%%%%%%%%%%%%%%%%%%%%%%%%%%%%%%%%%%%%%%%%%%%%%%%%%%%%
%
\def\tfrac#1#2{{\textstyle {#1 \over #2}}}%
\def\dfrac#1#2{{\displaystyle {#1 \over #2}}}%
\def\binom#1#2{{#1 \choose #2}}%
\def\tbinom#1#2{{\textstyle {#1 \choose #2}}}%
\def\dbinom#1#2{{\displaystyle {#1 \choose #2}}}%
\def\QATOP#1#2{{#1 \atop #2}}%
\def\QTATOP#1#2{{\textstyle {#1 \atop #2}}}%
\def\QDATOP#1#2{{\displaystyle {#1 \atop #2}}}%
\def\QABOVE#1#2#3{{#2 \above#1 #3}}%
\def\QTABOVE#1#2#3{{\textstyle {#2 \above#1 #3}}}%
\def\QDABOVE#1#2#3{{\displaystyle {#2 \above#1 #3}}}%
\def\QOVERD#1#2#3#4{{#3 \overwithdelims#1#2 #4}}%
\def\QTOVERD#1#2#3#4{{\textstyle {#3 \overwithdelims#1#2 #4}}}%
\def\QDOVERD#1#2#3#4{{\displaystyle {#3 \overwithdelims#1#2 #4}}}%
\def\QATOPD#1#2#3#4{{#3 \atopwithdelims#1#2 #4}}%
\def\QTATOPD#1#2#3#4{{\textstyle {#3 \atopwithdelims#1#2 #4}}}%
\def\QDATOPD#1#2#3#4{{\displaystyle {#3 \atopwithdelims#1#2 #4}}}%
\def\QABOVED#1#2#3#4#5{{#4 \abovewithdelims#1#2#3 #5}}%
\def\QTABOVED#1#2#3#4#5{{\textstyle 
   {#4 \abovewithdelims#1#2#3 #5}}}%
\def\QDABOVED#1#2#3#4#5{{\displaystyle 
   {#4 \abovewithdelims#1#2#3 #5}}}%
%
% Macros for text size operators:

%JCS - added braces and \mathop around \displaystyle\int, etc.
%
\def\tint{\mathop{\textstyle \int}}%
\def\tiint{\mathop{\textstyle \iint }}%
\def\tiiint{\mathop{\textstyle \iiint }}%
\def\tiiiint{\mathop{\textstyle \iiiint }}%
\def\tidotsint{\mathop{\textstyle \idotsint }}%
\def\toint{\mathop{\textstyle \oint}}%
\def\tsum{\mathop{\textstyle \sum }}%
\def\tprod{\mathop{\textstyle \prod }}%
\def\tbigcap{\mathop{\textstyle \bigcap }}%
\def\tbigwedge{\mathop{\textstyle \bigwedge }}%
\def\tbigoplus{\mathop{\textstyle \bigoplus }}%
\def\tbigodot{\mathop{\textstyle \bigodot }}%
\def\tbigsqcup{\mathop{\textstyle \bigsqcup }}%
\def\tcoprod{\mathop{\textstyle \coprod }}%
\def\tbigcup{\mathop{\textstyle \bigcup }}%
\def\tbigvee{\mathop{\textstyle \bigvee }}%
\def\tbigotimes{\mathop{\textstyle \bigotimes }}%
\def\tbiguplus{\mathop{\textstyle \biguplus }}%
%
%
%Macros for display size operators:
%

\def\dint{\mathop{\displaystyle \int}}%
\def\diint{\mathop{\displaystyle \iint }}%
\def\diiint{\mathop{\displaystyle \iiint }}%
\def\diiiint{\mathop{\displaystyle \iiiint }}%
\def\didotsint{\mathop{\displaystyle \idotsint }}%
\def\doint{\mathop{\displaystyle \oint}}%
\def\dsum{\mathop{\displaystyle \sum }}%
\def\dprod{\mathop{\displaystyle \prod }}%
\def\dbigcap{\mathop{\displaystyle \bigcap }}%
\def\dbigwedge{\mathop{\displaystyle \bigwedge }}%
\def\dbigoplus{\mathop{\displaystyle \bigoplus }}%
\def\dbigodot{\mathop{\displaystyle \bigodot }}%
\def\dbigsqcup{\mathop{\displaystyle \bigsqcup }}%
\def\dcoprod{\mathop{\displaystyle \coprod }}%
\def\dbigcup{\mathop{\displaystyle \bigcup }}%
\def\dbigvee{\mathop{\displaystyle \bigvee }}%
\def\dbigotimes{\mathop{\displaystyle \bigotimes }}%
\def\dbiguplus{\mathop{\displaystyle \biguplus }}%
%
%Companion to stackrel
\def\stackunder#1#2{\mathrel{\mathop{#2}\limits_{#1}}}%
%
%
% These are AMS environments that will be defined to
% be verbatims if amstex has not actually been 
% loaded
%
%
\begingroup \catcode `|=0 \catcode `[= 1
\catcode`]=2 \catcode `\{=12 \catcode `\}=12
\catcode`\\=12 
|gdef|@alignverbatim#1\end{align}[#1|end[align]]
|gdef|@salignverbatim#1\end{align*}[#1|end[align*]]

|gdef|@alignatverbatim#1\end{alignat}[#1|end[alignat]]
|gdef|@salignatverbatim#1\end{alignat*}[#1|end[alignat*]]

|gdef|@xalignatverbatim#1\end{xalignat}[#1|end[xalignat]]
|gdef|@sxalignatverbatim#1\end{xalignat*}[#1|end[xalignat*]]

|gdef|@gatherverbatim#1\end{gather}[#1|end[gather]]
|gdef|@sgatherverbatim#1\end{gather*}[#1|end[gather*]]

|gdef|@gatherverbatim#1\end{gather}[#1|end[gather]]
|gdef|@sgatherverbatim#1\end{gather*}[#1|end[gather*]]


|gdef|@multilineverbatim#1\end{multiline}[#1|end[multiline]]
|gdef|@smultilineverbatim#1\end{multiline*}[#1|end[multiline*]]

|gdef|@arraxverbatim#1\end{arrax}[#1|end[arrax]]
|gdef|@sarraxverbatim#1\end{arrax*}[#1|end[arrax*]]

|gdef|@tabulaxverbatim#1\end{tabulax}[#1|end[tabulax]]
|gdef|@stabulaxverbatim#1\end{tabulax*}[#1|end[tabulax*]]


|endgroup
  

  
\def\align{\@verbatim \frenchspacing\@vobeyspaces \@alignverbatim
You are using the "align" environment in a style in which it is not defined.}
\let\endalign=\endtrivlist
 
\@namedef{align*}{\@verbatim\@salignverbatim
You are using the "align*" environment in a style in which it is not defined.}
\expandafter\let\csname endalign*\endcsname =\endtrivlist




\def\alignat{\@verbatim \frenchspacing\@vobeyspaces \@alignatverbatim
You are using the "alignat" environment in a style in which it is not defined.}
\let\endalignat=\endtrivlist
 
\@namedef{alignat*}{\@verbatim\@salignatverbatim
You are using the "alignat*" environment in a style in which it is not defined.}
\expandafter\let\csname endalignat*\endcsname =\endtrivlist




\def\xalignat{\@verbatim \frenchspacing\@vobeyspaces \@xalignatverbatim
You are using the "xalignat" environment in a style in which it is not defined.}
\let\endxalignat=\endtrivlist
 
\@namedef{xalignat*}{\@verbatim\@sxalignatverbatim
You are using the "xalignat*" environment in a style in which it is not defined.}
\expandafter\let\csname endxalignat*\endcsname =\endtrivlist




\def\gather{\@verbatim \frenchspacing\@vobeyspaces \@gatherverbatim
You are using the "gather" environment in a style in which it is not defined.}
\let\endgather=\endtrivlist
 
\@namedef{gather*}{\@verbatim\@sgatherverbatim
You are using the "gather*" environment in a style in which it is not defined.}
\expandafter\let\csname endgather*\endcsname =\endtrivlist


\def\multiline{\@verbatim \frenchspacing\@vobeyspaces \@multilineverbatim
You are using the "multiline" environment in a style in which it is not defined.}
\let\endmultiline=\endtrivlist
 
\@namedef{multiline*}{\@verbatim\@smultilineverbatim
You are using the "multiline*" environment in a style in which it is not defined.}
\expandafter\let\csname endmultiline*\endcsname =\endtrivlist


\def\arrax{\@verbatim \frenchspacing\@vobeyspaces \@arraxverbatim
You are using a type of "array" construct that is only allowed in AmS-LaTeX.}
\let\endarrax=\endtrivlist

\def\tabulax{\@verbatim \frenchspacing\@vobeyspaces \@tabulaxverbatim
You are using a type of "tabular" construct that is only allowed in AmS-LaTeX.}
\let\endtabulax=\endtrivlist

 
\@namedef{arrax*}{\@verbatim\@sarraxverbatim
You are using a type of "array*" construct that is only allowed in AmS-LaTeX.}
\expandafter\let\csname endarrax*\endcsname =\endtrivlist

\@namedef{tabulax*}{\@verbatim\@stabulaxverbatim
You are using a type of "tabular*" construct that is only allowed in AmS-LaTeX.}
\expandafter\let\csname endtabulax*\endcsname =\endtrivlist

% macro to simulate ams tag construct


% This macro is a fix to eqnarray
\def\@@eqncr{\let\@tempa\relax
    \ifcase\@eqcnt \def\@tempa{& & &}\or \def\@tempa{& &}%
      \else \def\@tempa{&}\fi
     \@tempa
     \if@eqnsw
        \iftag@
           \@taggnum
        \else
           \@eqnnum\stepcounter{equation}%
        \fi
     \fi
     \global\tag@false
     \global\@eqnswtrue
     \global\@eqcnt\z@\cr}


% This macro is a fix to the equation environment
 \def\endequation{%
     \ifmmode\ifinner % FLEQN hack
      \iftag@
        \addtocounter{equation}{-1} % undo the increment made in the begin part
        $\hfil
           \displaywidth\linewidth\@taggnum\egroup \endtrivlist
        \global\tag@false
        \global\@ignoretrue   
      \else
        $\hfil
           \displaywidth\linewidth\@eqnnum\egroup \endtrivlist
        \global\tag@false
        \global\@ignoretrue 
      \fi
     \else   
      \iftag@
        \addtocounter{equation}{-1} % undo the increment made in the begin part
        \eqno \hbox{\@taggnum}
        \global\tag@false%
        $$\global\@ignoretrue
      \else
        \eqno \hbox{\@eqnnum}% $$ BRACE MATCHING HACK
        $$\global\@ignoretrue
      \fi
     \fi\fi
 } 

 \newif\iftag@ \tag@false
 
 \def\tag{\@ifnextchar*{\@tagstar}{\@tag}}
 \def\@tag#1{%
     \global\tag@true
     \global\def\@taggnum{(#1)}}
 \def\@tagstar*#1{%
     \global\tag@true
     \global\def\@taggnum{#1}%  
}

% Do not add anything to the end of this file.  
% The last section of the file is loaded only if 
% amstex has not been.



\makeatother
\endinput

\begin{document}


\subsection{Model Assumptions}

The world economy is composed of $N$ countries. At a given point in time $t$%
, each country $n$ is endowed with $L_{nt}$ units of a primary (non
produced) input, which we interpret as equipped labour. There are $J$
sectors (or broad classes of goods) in the economy, whose output is combined
into a final good through a Cobb-Douglas aggregate. In formulas, aggregate
gross output in the economy is given by: 
\begin{equation}
Q_{nt}=\prod_{j=1}^{J}\left( Q_{nt}^{j}\right) ^{\alpha ^{j}}
\label{aggregate}
\end{equation}%
where $Q_{t}^{j}$ is the gross output in sector $j$ and $\sum_{j=1}^{J}%
\alpha ^{j}=1$. Competitive firms in each sector $j$ produce a composite
good according to the following constant-elasticity-of-substitution (CES)
technology: 
\begin{equation}
Q_{nt}^{j}=\left[ \int_{0}^{1}q_{nt}(\omega ^{j})^{\frac{\eta -1}{\eta }%
}d\omega ^{j}\right] ^{\frac{\eta }{\eta -1}}  \label{sectoroutput}
\end{equation}%
where $q_{nt}(\omega ^{j})$ is the quantity of good $\omega ^{j}$ used by
country $n$ in sector $j$ at time $t$, and $\eta >0$ is the elasticity of
substitution across goods within a given sector. The intermediate goods $%
\omega ^{j}$ can be produced locally or imported from other countries.
Delivering a good from country $n$ to country $m$ in sector $j$ and time
period $t$ results in $0<\kappa _{mnt}^{j}\leq 1$ goods arriving at $m$; we
assume that $\kappa _{mnt}^{j}\geq \kappa _{mkt}^{j}\kappa _{knt}^{j}\quad
\forall m,n,k,j,t$ and $\kappa _{nnt}^{j}=1$. All costs incurred are net
losses.\footnote{%
In the calibration, the $\kappa $s will reflect all trading costs, including
tariffs; so implicitly we adopt the extreme assumption that tariff revenues
are wasted---or at least not rebated back to agents in a way that would
interact with the allocation of resources in the economy.} Under the
assumption of perfect competition, goods are sourced from the lowest-cost
producer, after adjusting for transport costs. The technology for producing $%
q_{nt}(\omega ^{j})$ is given accordingly by the country of origin ($m$)
with the lowest cost (with $m=n$ when the good is produced locally): 
\begin{equation}
x_{mt}(\omega ^{j})=A_{mt}^{j}z_{m}(\omega ^{j})l_{mt}(\omega ^{j})^{\beta
^{j}}M_{mt}(\omega ^{j})^{1-\beta ^{j}}
\end{equation}%
where $x_{mt}(\omega ^{j})$ is the production of good $\omega ^{j}$ by
country $m$ at time $t$, $M_{mt}(\omega ^{j})$ is the amount of the
aggregate composite good used by country $m$ to produce $x_{mt}(\omega ^{j})$
units of good $\omega ^{j}$ and $l_{mt}(\omega ^{j})$ is the corresponding
amount of equipped labour. Total factor productivity (TFP) levels vary
across countries, sectors, and goods. Specifically, each intermediate good $%
\omega ^{j}$ in sector $j$ of country $n$ has a time-invariant idiosyncratic
productivity factor $z_{n}(\omega ^{j})$ and a time-varying factor $%
A_{nt}^{j}$ common to all the goods $\omega ^{j}$ in sector $j$. Building on
the literature, we assume the productivities $z_{n}(\omega ^{j})$ follow a
sector-specific, time-invariant Fr\'{e}chet distribution $F_{n}^{j}(z)=\exp
(-T_{n}^{j}z^{-\theta })$. A higher $T_{n}^{j}$ shifts the distribution of
productivities to the right, that is leading to probabilistically higher
productivities. A higher $\theta $ decreases the dispersion of the
productivity distribution, and hence reduces the scope for comparative
advantage. Shocks to $A_{nt}^{j}$ over time are interpreted as standard
sectoral total factor productivity (TFP) shocks.

The single final good can be used both as input in the production of
intermediaries $\omega^{j}$ or for final consumption, $C_{nt}$. Hence,
market clearing in the good markets implies: 
\begin{equation*}
Q_{nt} = C_{nt} + \sum_{j=1}^{J}\int_{0}^{1}M_{nt}(\omega^{j})d\omega ^{j}, 
\end{equation*}
where the integral aggregates over the unit-size continuum of goods $%
\omega^{j}$ entering in the production of each sector's $j$ aggregate good.

Clearing in the input market within a sector implies: 
\begin{equation*}
L_{nt}^{j}=\int_{0}^{1}l_{nt}(\omega ^{j})d\omega ^{j}, 
\end{equation*}%
where $l_{nt}(\omega ^{j})$ denotes the amount of equipped labour used in
the production of good $\omega ^{j}$ by country $n$. We assume there is
perfect risk-sharing within a country, but no risk-sharing across countries.%
\footnote{%
To motivate the lack of risk-sharing across countries, see our discussion of
Figures 1 and 2.} The (equipped) labour shares allocated to each sector, $%
L_{nt}^{j}/L_{nt},$ with $\sum\nolimits_{k=1}^{J}L_{nt}^{k}=L_{nt}$, are
determined ex ante (before the realization of the shocks), but can be
adjusted ex-post, at an adjustment cost. At the beginning of each period, a
representative consumer decides on the ex-ante optimal allocation of the
primary input $L_{nt}$ into different sectors in order to maximize the
expected value of utility. Then (stochastic) shocks to productivity $%
A_{nt}^{j}$ are realized; equipped labour is reallocated freely within a
sector, while reallocation across sectors implies a disutility cost $%
\Upsilon .$ Production and consumption take place. The disutility cost of
ex-post reallocation across sectors aims at capturing the idea that in the
short run, it is costly to reallocate productive factors across sectors.

The representative consumer's budget constraint in each period is: 
\begin{equation*}
P_{nt}C_{nt}=\sum_{j=1}^{J}w_{nt}^{j}L_{nt}^{j}, 
\end{equation*}%
where $P_{nt}$ is the price of the aggregate good (\ref{aggregate}), $%
w_{nt}^{j}L_{nt}^{j}$ is the nominal value-added generated in sector $j$.
Lifetime utility is given by 
\begin{equation*}
U_{n}=\sum\limits_{t=0}^{\infty }\delta ^{t}u(C_{nt})+\Upsilon , 
\end{equation*}%
where $u^{\prime }>0$, $u^{\prime \prime }\leq 0$ and $\delta $ is the
discount factor and $\Upsilon $ is the sum of quadratic deviations of
sectoral shares from the original sectoral allocation. The cost $\Upsilon $
is relevant ex-post, when consumers need to reallocate inputs across
sectors. Ex ante, however, the optimal allocation is such that the expected
adjustment costs will be zero. Because there is no intertemporal trade and
no capital in the economy, and given that input shares ex ante will equal
the expected optimal allocation, each period consumers maximize the expected
static utility flow $E\left[ u\left( C_{nt}\right) \right] $ and the
equilibrium is simply a sequence of static equilibria. In making his labor
allocation decisions the representative consumer takes into account the
joint probability distribution function of sectoral productivities, $%
A_{nt}^{j}$s.

In the analysis, we assume log utility and therefore the consumer solves: 
\begin{equation}
L_{nt}^{k\ast }=\arg \max E_{t-1}\left[ \ln \left( \frac{%
\sum_{j=1}^{J}w_{nt}^{j}L_{nt}^{j}}{P_{nt}}\right) \right]
,s.t.:\sum\nolimits_{j=1}^{J}L_{nt}^{j}=L_{nt},  \label{eq:log:utility}
\end{equation}%
where $E_{t-1}$ indicates that the expectation is taken before the
realization of period $t$ shocks. This\ maximization problem leads to the
following first-order conditions for the ex-ante allocation of inputs to
sectors: 
\begin{equation}
\psi _{nt}^{k\ast }=\frac{L_{nt}^{k\ast }}{L_{nt}}=E_{t-1}\left[ \frac{%
w_{nt}^{k}L_{nt}^{k}}{\sum_{k}w_{nt}^{k}L_{nt}^{k}}\right] ,\qquad \forall
k,t.  \label{eq:FOC}
\end{equation}%
In words, the share of resources allocated ex ante to a given sector equals
its expected share in value added. To gain intuition on this expression note
that $1/\sum_{k}w_{nt}^{k}L_{nt}^{k}$ is the marginal utility of consumption
in period $t$; thus, more resources are allocated to higher value-added
sectors, after appropriately weighting by marginal utility. Consider, for
further intuition, a (small) sector whose productivity is negatively
correlated with the rest of the economy (that is, it has high value added
when the rest of the economy has low value added); in states of the world in
which overall income is low, the marginal utility of consumption $%
1/\sum_{k}w_{nt}^{k}L_{nt}^{k}$ will be high and hence the optimal
allocation entails allocating more resources to this sector. In the closed
economy, the value-added share is pinned down by the Cobb--Douglas
coefficients $\alpha ^{j}\beta ^{j}$, as with Cobb-Douglas technology there
is no variation on expenditures (and sales) shares---and log-utility implies
the shares determine the sectoral allocation of resources. In the open
economy this result no longer holds as a country's sectoral shares depend on
its absolute and comparative advantage as well as trading costs vis-\`{a}%
-vis other countries.

Ex post, as said, labour can potentially be reallocated after incurring the
ajustment costs. The ex-post sectoral input allocation solves:

\begin{equation}
L_{nt}^{k}=\arg \max \left[ \ln \left( \frac{%
\sum_{j=1}^{J}w_{nt}^{j}L_{nt}^{j}}{P_{nt}}\right) +\frac{\varrho }{2}%
\sum_{j=1}^{J}\left[ \psi _{nt}^{j}-\psi _{nt}^{j\ast }\right] ^{2}\right] ,%
\text{ }s.t.:\sum\nolimits_{j=1}^{J}L_{nt}^{j}=L_{nt}
\end{equation}%
with $\psi _{nt}^{j}\equiv \frac{L_{nt}^{j}}{L_{nt}}$ and $%
\sum_{j=1}^{J}\psi _{nt}^{j}=\sum_{j=1}^{J}\psi _{nt}^{j\ast }=1.$ The first
order conditions satisfy%
\begin{equation}
\frac{w_{nt}^{k}}{\sum_{j=1}^{J}w_{nt}^{j}L_{nt}^{j}}+\varrho \left[ \psi
_{nt}^{k}-\psi _{nt}^{k\ast }\right] =\lambda
\end{equation}%
where $\lambda $ is given by $\lambda =\frac{\frac{1}{J}%
\sum_{j=1}^{J}w_{nt}^{j}}{\sum_{j=1}^{J}w_{nt}^{j}L_{nt}^{j}}$ and hence

\begin{equation*}
\psi _{nt}^{k}=\psi _{nt}^{k\ast }+\frac{1}{\varrho }\left[ \frac{\frac{1}{J}%
\sum_{j=1}^{J}w_{nt}^{j}-w_{nt}^{k}}{\sum_{j=1}^{J}w_{nt}^{j}L_{nt}^{j}}%
\right] 
\end{equation*}%
In words, the ex post input shares $\psi _{nt}^{k}$ equal the ex-ante
optimal shares $\psi _{nt}^{k\ast }$ plus a fraction of the differential
between the average equipped labour cost in the economy $\frac{1}{J}%
\sum_{j=1}^{J}w_{nt}^{j}$ and the sectoral input cost $w_{nt}^{k}$. The
adjustment cost parameter $\varrho $ determines the elasticity of sectoral
adjustment to the cost differential. In the extreme, when adjustment costs
are infinite, $\varrho \rightarrow \infty $, the economy simply stays at the
initial sectoral input shares $\psi _{nt}^{k}=\psi _{nt}^{k\ast }$. As
adjustment costs decrease, the sectoral shares adjust in response to the
cost differential.

\bigskip

PAUSE

\subsection{Model Solution}

We first discuss the solution under autarky, and then turn to the solution
under free trade.

\paragraph{Solution under Autarky}

We solve the model backwards in two stages. First, we solve the model taking
the sectoral allocation of nonproduced inputs $L_{t}^{j}$ as fixed. We then
solve for the ex-ante optimal $L_{t}^{j\prime }s$ before the shocks are
realized. In the analysis of the autarky case, we omit the country-specific
subscripts $n$ for convenience.

The demand for each sector's composite good is given by 
\begin{equation}
Q_{t}^{j}=\alpha _{t}^{j}\left( \frac{P_{t}^{j}}{P_{t}}\right) ^{-1}Q_{t},
\label{QJ}
\end{equation}%
and the demand for each intermediate good $\omega ^{j}$ is 
\begin{equation*}
q_{t}(\omega ^{j})=\left[ \frac{p_{t}(\omega ^{j})}{P_{t}^{j}}\right]
^{-\eta }Q_{t}^{j}, 
\end{equation*}%
where 
\begin{equation}
P_{t}^{j}=\left[ \int_{0}^{1}p_{t}(\omega ^{j})^{1-\eta }d\omega ^{j}\right]
^{\frac{1}{1-\eta }}  \label{PJ}
\end{equation}%
is the aggregate price index in sector $j$, and the economy-wide price index
is given by: 
\begin{equation}
P_{t}=\prod_{j=1}^{J}\alpha _{t}^{j^{-\alpha _{t}^{j}}}\left(
P_{t}^{j}\right) ^{\alpha _{t}^{j}}.  \label{Pt}
\end{equation}%
The demand for non-produced inputs $l_{t}(\omega ^{j})$ and produced inputs $%
M_{t}(\omega ^{j})$ are given, respectively, by $l_{t}^{j}(\omega
^{j})=\beta ^{j}\frac{p_{t}(\omega ^{j})q_{t}(\omega ^{j})}{w_{t}^{j}}$ and $%
M_{t}(\omega ^{j})=(1-\beta ^{j})\frac{p_{t}(\omega ^{j})q_{t}(\omega ^{j})}{%
P_{t}}.$ Aggregating over all goods $\omega ^{j}$ in a given sector, we
obtain 
\begin{equation}
w_{t}^{j}L_{t}^{j}=\beta ^{j}P_{t}^{j}Q_{t}^{j}  \label{Labor share}
\end{equation}%
and, correspondingly, $P_{t}M_{t}^{j}=(1-\beta ^{j})P_{t}^{j}Q_{t}^{j}.$
Labour shares are given by:

\begin{equation*}
\frac{L_{t}^{j}}{L_{t}}=\psi _{nt}^{k}=\psi _{nt}^{k\ast }+\frac{1}{\varrho }%
\left[ \frac{\frac{1}{J}\sum_{j=1}^{J}w_{nt}^{j}-w_{nt}^{k}}{%
\sum_{j=1}^{J}w_{nt}^{j}L_{nt}^{j}}\right] 
\end{equation*}%
and the initial labour allocation, $\psi _{nt}^{k\ast }$, is taken as given.

Using the input demand functions and the zero profit condition the autarky
prices of intermediate goods are given by: 
\begin{equation}
p_{t}(\omega ^{j})=B^{j}\left[ A_{t}^{j}\cdot z(\omega ^{j})\right]
^{-1}\left( w_{t}^{j}\right) ^{\beta ^{j}}P_{t}^{1-\beta ^{j}},
\label{Frechet}
\end{equation}%
where $B^{j}=$ $\left( \beta ^{j}\right) ^{-\beta ^{j}}(1-\beta
^{j})^{-(1-\beta ^{j})}$. Using (\ref{Frechet}) and the properties of the Fr%
\'{e}chet distribution, we can express the sectoral price index as: 
\begin{equation}
P_{t}^{j}=\xi B^{j}\left[ A_{t}^{j}\cdot \left( T^{j}\right) ^{\frac{1}{%
\theta }}\right] ^{-1}\left( w_{t}^{j}\right) ^{\beta ^{j}}P_{t}^{1-\beta
^{j}}
\end{equation}%
where $\xi =\left[ \Gamma \left( \frac{\theta +1-\eta }{\theta }\right) %
\right] ,$ and $\Gamma $ is the gamma function.

Using (\ref{QJ}), (\ref{Labor share}), and (\ref{Pt}) we obtain real GDP:

\begin{eqnarray*}
Y_{t} &=&\sum_{j=1}^{J}\frac{w_{t}^{j}L_{t}^{j}}{P_{t}}=\sum_{j=1}^{J}\beta
^{j}\frac{P_{t}^{j}Q_{t}^{j}}{P_{t}}=\sum_{j=1}^{J}\beta ^{j}\alpha
_{t}^{j}Q_{t} \\
\text{since }w_{t}^{j}L_{t}^{j} &=&\beta ^{j}P_{t}^{j}Q_{t}^{j}=\beta
^{j}\alpha _{t}^{j}P_{t}Q_{t} \\
Q_{t} &=&\frac{1}{\alpha _{t}^{j}\beta ^{j}}\frac{w_{t}^{j}}{P_{t}}L_{t}^{j}
\\
\text{and since }\frac{w_{t}^{j}}{P_{t}} &=&\left( \xi B^{j}\right)
^{-1/\beta ^{j}}\left[ A_{t}^{j}\cdot \left( T^{j}\right) ^{\frac{1}{\theta }%
}\right] ^{1/\beta ^{j}}\left( \frac{P_{t}^{j}}{P_{t}}\right) ^{\frac{1}{%
\beta ^{j}}}
\end{eqnarray*}%
We hence obtain%
\begin{eqnarray*}
Q_{t} &=&\left( \xi B^{j}\right) ^{-1/\beta ^{j}}\left[ A_{t}^{j}\cdot
\left( T^{j}\right) ^{\frac{1}{\theta }}\right] ^{1/\beta ^{j}}\left( \frac{%
P_{t}^{j}}{P_{t}}\right) ^{\frac{1}{\beta ^{j}}}\frac{1}{\alpha
_{t}^{j}\beta ^{j}}L_{t}^{j} \\
\frac{P_{t}^{j}}{P_{t}} &=&\left( \xi B^{j}\right) \left[ A_{t}^{j}\cdot
\left( T^{j}\right) ^{\frac{1}{\theta }}\right] ^{-1}\left( \frac{1}{\alpha
_{t}^{j}\beta ^{j}}L_{t}^{j}\right) ^{-\beta ^{j}}\left( Q_{t}\right)
^{\beta ^{j}} \\
\prod_{j=1}^{J}\alpha _{t}^{j^{-\alpha _{t}^{j}}}\left( P_{t}^{j}\right)
^{\alpha _{t}^{j}} &=&\prod_{j=1}^{J}\alpha _{t}^{j^{-\alpha
_{t}^{j}}}\left( \xi B^{j}\right) ^{\alpha _{t}^{j}}\left[ A_{t}^{j}\cdot
\left( T^{j}\right) ^{\frac{1}{\theta }}\right] ^{-\alpha _{t}^{j}}\left( 
\frac{1}{\alpha _{t}^{j}\beta ^{j}}L_{t}^{j}\right) ^{-\beta ^{j}\alpha
_{t}^{j}}\left( Q_{t}\right) ^{\beta ^{j}\alpha _{t}^{j}} \\
1 &=&\prod_{j=1}^{J}\alpha _{t}^{j^{-\alpha _{t}^{j}}}\left( \xi
B^{j}\right) ^{\alpha _{t}^{j}}\left[ A_{t}^{j}\cdot \left( T^{j}\right) ^{%
\frac{1}{\theta }}\right] ^{-\alpha _{t}^{j}}\left( \frac{1}{\alpha
_{t}^{j}\beta ^{j}}L_{t}^{j}\right) ^{-\beta ^{j}\alpha _{t}^{j}}Q_{t}^{\bar{%
\beta}} \\
Q_{t} &=&\prod_{j=1}^{J}\left( \xi B^{j}\right) ^{-\alpha ^{j}/\bar{\beta}}%
\left[ A_{t}^{j}\cdot \left( T^{j}\right) ^{\frac{1}{\theta }}\right]
^{\alpha ^{j}/\bar{\beta}}\alpha _{t}^{j^{-\alpha _{t}^{j}/\bar{\beta}%
}}\left( \alpha _{t}^{j}\beta ^{j}\right) ^{-\frac{\beta ^{j}\alpha _{t}^{j}%
}{\bar{\beta}}}\left( L_{t}^{j}\right) ^{\frac{\beta ^{j}\alpha _{t}^{j}}{%
\bar{\beta}}} \\
Y_{t} &=&\sum_{j=1}^{J}\frac{w_{t}^{j}L_{t}^{j}}{P_{t}}=\sum_{k=1}^{J}\beta
^{k}\alpha _{t}^{k}Q_{t}=\bar{\beta}\prod_{j=1}^{J}\left( \xi B^{j}\right)
^{-\alpha ^{j}/\bar{\beta}}\left[ A_{t}^{j}\cdot \left( T^{j}\right) ^{\frac{%
1}{\theta }}\right] ^{\alpha ^{j}/\bar{\beta}}\alpha _{t}^{j^{-\alpha
_{t}^{j}/\bar{\beta}}}\left( \alpha _{t}^{j}\beta ^{j}\right) ^{-\frac{\beta
^{j}\alpha _{t}^{j}}{\bar{\beta}}}\left( L_{t}^{j}\right) ^{\frac{\beta
^{j}\alpha _{t}^{j}}{\bar{\beta}}} \\
Y_{t} &=&\prod_{j=1}^{J}R^{j}\left[ A_{t}^{j}\cdot \left( T^{j}\right) ^{%
\frac{1}{\theta }}\right] ^{\alpha ^{j}/\bar{\beta}}\left( L_{t}^{j}\right)
^{\frac{\beta ^{j}\alpha _{t}^{j}}{\bar{\beta}}}
\end{eqnarray*}

\bigskip where $\bar{\beta}=\sum_{j=1}^{J}\alpha ^{j}\beta ^{j}$ and $%
R_{j}\propto \prod_{j=1}^{J}\left( \beta ^{j}\alpha ^{j}\right) ^{-\frac{%
\alpha ^{j}\beta ^{j}}{\bar{\beta}}}(B^{j})^{-\frac{\alpha ^{j}}{\bar{\beta}}%
}\left( T^{j}\right) ^{\frac{\alpha ^{j}}{\bar{\beta}\cdot \theta }}$ is a
time-invariant product. We now need to determine $L_{nt}^{k}$

\begin{equation}
\psi _{nt}^{k\ast }=\frac{L_{nt}^{k\ast }}{L_{nt}}=E_{t-1}\left[ \frac{%
w_{nt}^{k}L_{nt}^{k}}{\sum_{k}w_{nt}^{k}L_{nt}^{k}}\right] ,\qquad \forall
k,t.
\end{equation}%
\begin{equation*}
\psi _{nt}^{k}=\psi _{nt}^{k\ast }+\frac{1}{\varrho }\left[ \frac{\frac{1}{J}%
\sum_{j=1}^{J}w_{nt}^{j}-w_{nt}^{k}}{\sum_{j=1}^{J}w_{nt}^{j}L_{nt}^{j}}%
\right] 
\end{equation*}

We can now move one step backward and solve for the allocation of the
primary input across sectors, $L_{t}^{j},$ $j=1,...,J$. 

$\psi _{nt}^{k\ast }=\frac{\alpha ^{j}\beta ^{j}}{\bar{\beta}}$.$.$Hence:

\begin{equation*}
\psi _{t}^{k}=\frac{\alpha ^{j}\beta ^{j}}{\bar{\beta}}+\frac{1}{\varrho }%
\left[ \frac{\bar{w}_{t}-w_{t}^{k}}{\sum_{j=1}^{J}w_{t}^{j}L_{t}^{j}}\right] 
\end{equation*}

\begin{eqnarray}
Y_{t} &=&\sum_{j=1}^{J}\frac{w_{t}^{j}L_{t}^{j}}{P_{t}}=\prod_{j=1}^{J}R_{j}%
\left[ A_{t}^{j}\right] ^{\frac{\alpha ^{j}}{\bar{\beta}}}\left( \frac{%
\alpha ^{j}\beta ^{j}}{\bar{\beta}}+\frac{1}{\varrho }\left[ \frac{\bar{w}%
_{t}-w_{t}^{k}}{\sum_{j=1}^{J}w_{t}^{j}L_{t}^{j}}\right] \right) ^{\frac{%
\alpha ^{j}\beta ^{j}}{\bar{\beta}}}L_{t}  \label{expost} \\
Y_{t} &=&\sum_{j=1}^{J}\frac{w_{t}^{j}L_{t}^{j}}{P_{t}}=\prod_{j=1}^{J}R_{j}%
\left[ A_{t}^{j}\right] ^{\frac{\alpha ^{j}}{\bar{\beta}}}\left( \frac{%
\alpha ^{j}\beta ^{j}}{\bar{\beta}}+\frac{1}{\varrho }\left[ \frac{\frac{%
\bar{w}_{t}-w_{t}^{k}}{P_{t}}}{Y_{t}}\right] \right) ^{\frac{\alpha
^{j}\beta ^{j}}{\bar{\beta}}}L_{t}
\end{eqnarray}

\begin{equation*}
\frac{\bar{w}_{t}-w_{t}^{k}}{P_{t}}=\sum_{j\neq k}^{J}\left( \xi
B^{j}\right) ^{-1/\beta ^{j}}\left[ A_{t}^{j}\cdot \left( T^{j}\right) ^{%
\frac{1}{\theta }}\right] ^{1/\beta ^{j}}\left( \frac{P_{t}^{j}}{P_{t}}%
\right) ^{\frac{1}{\beta ^{j}}}
\end{equation*}

\bigskip 

\bigskip 

\bigskip 

STOP HERE

\bigskip

\paragraph{Solution with International Trade}

The key difference in the internationally open economy is that inputs can
potentially be sourced from different countries. Delivering a unit of good $%
\omega ^{j}$ produced in country $m$ to country $n$ costs: 
\begin{equation*}
p_{nmt}^{j}(\omega ^{j})=\frac{B^{j}\left( w_{mt}^{j}\right) ^{\beta
^{j}}P_{mt}^{1-\beta ^{j}}}{A_{mt}^{j}\kappa _{nmt}^{j}z_{m}(\omega ^{j})} 
\end{equation*}%
where $B^{j}\left( w_{mt}^{j}\right) ^{\beta ^{j}}P_{mt}^{1-\beta ^{j}}$ is
the cost of the input bundle in country of origin $m$, sector $j$, at time $%
t $. Because of perfect competition, the price paid in country $n$, denoted $%
p_{nt}(\omega ^{j})$, will be the minimum price across all $N$ potential
trading partners: $p_{nt}^{j}(\omega ^{j})=\min \left\{ p_{nmt}^{j}(\omega
^{j});\text{ }m=1,...,N\right\} .$ Producers of the aggregate good in (\ref%
{aggregate}) minimize production costs taking prices as given. We assume the
distribution of efficiencies for any good $\omega ^{j}$ in sector $j$ and
country $n$ are independent across countries and sectors and follow a
time-invariant Fr\'{e}chet distribution: $F_{n}^{j}(z)=\exp
(-T_{n}^{j}z^{-\theta }).$ Under this assumption, the distribution of prices
in sector $j$ of country $n$, conditional on $\left\{ A_{mt}^{j}\right\}
_{m=1,...N}$ is given by $G_{nt}^{j}(p)|_{\left\{ A_{t}^{j}\right\} }=\Pr
(P_{nt}^{j}<p)=1-\exp \left[ -\Phi _{nt}^{j}p^{\theta }\right] $ where $\Phi
_{nt}^{j}=\sum_{m=1}^{N}T_{m}^{j}\left( \frac{B^{j}\left( w_{mt}^{j}\right)
^{\beta ^{j}}P_{mt}^{1-\beta ^{j}}}{A_{mt}^{j}\kappa _{nmt}^{j}}\right)
^{-\theta }$. Given that there is a continuum of $\omega ^{j}$ in each
sector, by the law of large numbers the probability that country $m$
provides a good in sector $j$ at the lowest price in country $n$ equals the
fraction of goods that country $n$ buys from country $m$ in sector $j$: 
\begin{equation}
d_{nmt}^{j}=\frac{T_{m}^{j}\left( \frac{B^{j}\left( w_{mt}^{j}\right)
^{\beta ^{j}}P_{mt}^{1-\beta ^{j}}}{A_{mt}^{j}\kappa _{nmt}^{j}}\right)
^{-\theta }}{\Phi _{nt}^{j}}  \label{shares}
\end{equation}%
that is, $d_{nmt}^{j}$ is the fraction of country $n$'s total spending on
sector-$j$ goods from country $m$ at time $t$. The equilibrium in the open
economy can be defined as following.

\textbf{Equilibrium Definition. } An equilibrium in the open economy is
defined as a set of resource allocations $\left\{ L_{nt}^{j}\right\} $,
import shares $\left\{ d_{nit}^{j}\right\} $, prices $\left\{ P_{nt}\right\} 
$, $\left\{ P_{nt}^{j}\right\} $, and $\left\{ w_{n}^{j}\right\} $ such
that, given technology $\left\{ A_{it}^{j}\right\}
,\left\{T_{it}^{j}\right\} ,$ aggregate endowments $\left\{ L_{nt}\right\} $
and trading costs $\left\{ \kappa _{int}^{j}\right\} $ $i)$ consumers
maximize expected utility, $ii)$ firms minimize costs and, $iii)$ markets
for goods and inputs clear, and $iv)$ trade is balanced. In equilibrium,
prices and quantities satisfy (\ref{eq1})-(\ref{eq7}): 
\begin{equation}
P_{nt}=\prod_{j}^{J}\left( \frac{1}{\alpha _{n}^{j}}\right)
^{\alpha^{j}}\left( P_{nt}^{j}\right) ^{\alpha ^{j}}  \label{eq1}
\end{equation}
\begin{equation}
P_{nt}^{j}=\xi\Phi _{nt}^{j^{-1/}\theta }  \label{eq2}
\end{equation}
\begin{equation}
\Phi _{nt}^{j}=\left( B^{j}\right) ^{-\theta
}\sum_{i=1}^{N}T_{i}^{j}\left(A_{it}^{j}\right) ^{\theta }\left[ \frac{%
P_{it}^{1-\beta ^{j}}\left(w_{it}^{j}\right) ^{\beta ^{j}}}{\kappa _{nit}^{j}%
}\right] ^{-\theta }  \label{eq3}
\end{equation}
\begin{equation}
d_{nmt}^{j}=\frac{\left( B^{j}\right) ^{-\theta
}T_{m}^{j}\left(A_{mt}^{j}\right) ^{\theta }\left( \frac{P_{mt}^{1-\beta
^{j}}w_{mt}^{j\beta^{j}}}{\kappa _{nmt}^{j}}\right) ^{-\theta }}{\Phi
_{nt}^{j}};\sum\limits_{m=1}^{N}d_{nmt}^{j}=1  \label{eq4}
\end{equation}
\begin{equation}
w_{nt}^{j}L_{nt}^{j}=\beta^{j}\sum_{m=1}^{N}d_{mnt}^{j}\left[\alpha^{j}+%
\frac{1-\beta^{j}}{\beta^j} \cdot \frac{w_{mt}^{j}L_{mt}^{j}}{w_{mt}L_{mt}}%
\right] w_{mt}L_{mt}  \label{eq5}
\end{equation}
\begin{equation}
w_{nt}L_{nt}=\sum_{j=1}^{J}w_{nt}^{j}L_{nt}^{j}  \label{eq6}
\end{equation}
\begin{equation}
\frac{L_{nt}^{j}}{L_{nt}}=E_{t-1}\left[ \frac{w_{nt}^{j}L_{nt}^{j}}{%
\sum_{k=1}^J w_{nt}^{k}L_{nt}^{k}}\right]  \label{eq7}
\end{equation}

Equations (\ref{eq1})--(\ref{eq3}) show the equilibrium prices as a function
of technology and input costs resulting from firms' cost minimization and
consumers' maximization problems. The first equation in (\ref{eq4}) shows
the value of goods from sector $j$ bought by country $n$ from country $m$ as
a share of total spending on goods $j$ by country $n.$ The second equation
says that the sum of spending shares on goods $j$ from all countries $m$ by
country $n$ (including $n$ itself) add to 1, that is, imports plus domestic
expenditures on goods $j$ by country $n$, add up to the overall spending
value on goods $j$ by country $n.$ Equation (\ref{eq5}) gives the value of
total sales accruing to the primitive factor in sector $j$ of country $n;$
it already incorporates the balanced trade condition, i.e., total payments
for goods flowing out of country $m$ to the rest of the world equal payments
flowing in country $m$ from the rest of the world.\footnote{%
In formulas, $\sum\nolimits_{j=1}^{J}X_{mt}^{j}=\sum\nolimits_{n=1}^{N}\sum%
\nolimits_{j=1}^{J}d_{nmt}^{j}X_{nt}^{j}$, where $X_{nt}^{j}$ is total
expenditure by country $n$ on sector-$j$ goods. The right-hand side is the
total demand by all $N$ countries for goods produced in country $m$. The
left-hand side is the total expenditures by country $m$, which, under trade
balance also equals its total sales. Recall that $P_{m}^{j}Q_{m}^{j}$ is the
total purchases of goods from sector $j$ by country $m.$ Note $%
P_{mt}^{j}Q_{mt}^{j}$, the total purchases of goods from sector $j$ by
country $m.$ Hence: $P_{mt}^{j}Q_{mt}^{j}=\alpha ^{j}w_{mt}L_{mt}+\frac{%
1-\beta ^{j}}{\beta ^{j}}w_{mt}^{j}L_{mt}^{j}$.} Equation (\ref{eq4})
expresses total value added in the economy as the sum of sectoral value
added. (Real value added is given by $Y_{nt}=\frac{w_{nt}L_{nt}}{P_{nt}}$.)
Finally, (\ref{eq7}) expresses the resource shares as a function of expected
shares, following the first order conditions in (\ref{eq:FOC}).

The model can conceptually be solved backwards in two steps. First, for any
given set of values for $L_{nt}^{j}$, the first five sets of equations can
be solved for $P_{nt}$, $w_{nt}^{j}$, $P_{nt}^{j}$, $d_{nmt}^{j}$ as a
function of the $\kappa _{mnt}^{j}s$ and the augmented productivity factors
defined as: 
\begin{equation}
Z_{nt}^{j}\equiv T_{n}^{j}\left[ L_{nt}\left( A_{nt}^{j}\right) ^{1/\beta
^{j}}\right] ^{\beta ^{j}\theta }.  \label{productivityfactor}
\end{equation}%
Then in a first stage, we can solve for the shares $\frac{L_{nt}^{j}}{L_{nt}}
$. As seen, with log utility the solution for $\frac{L_{n}^{j}}{L_{n}}$
simplifies significantly as it is the expected value of sectoral value-added
shares; in the implementation, we will use the data to help pin down these
expectations.

\subsection{Two Illustrative Cases: Autarky and Costless Trade}

To illustrate the mechanism of diversification through trade, we analyze a
one-sector version of the model (that is, the Eaton-Kortum model) under two
extreme cases for which we have closed-form analytical solutions for GDP:
autarky and costless trade. We accordingly drop the sector subscripts.

\subsubsection{Volatility under Autarky}

Under complete autarky, value added in the one-sector economy is given by (%
\ref{expost}), which can be rewritten as: 
\begin{equation*}
Y_{nt}\propto \left( Z_{nt}\right) ^{\frac{1}{\beta \theta }} 
\end{equation*}
where $Z_{nt}\equiv T_{n}\left( L_{nt}A_{nt}^{1/\beta }\right)^{\beta\theta
} $. Taking log-differences around the mean (or trend value in the
empirics), we obtain, 
\begin{equation*}
\hat{Y}_{nt}=\frac{1}{\beta \theta }\hat{Z}_{nt}. 
\end{equation*}
Thus, in the one-sector economy under autarky, shocks to value added are
driven exclusively by domestic shocks to the productive capacity of the
economy, $\hat{Z}_{nt}.$ The variance of GDP, $V(\hat{Y}_{nt})$ thus depends
on the variance of the shocks $V(\hat{Z}_{nt})$: 
\begin{equation*}
V(\hat{Y}_{nt})=\frac{1}{\left( \beta \theta \right) ^{2}}V(\hat{Z}_{nt}). 
\end{equation*}

\subsubsection{Volatility under Costless Trade}

Under costless trade in the one-sector economy ($\kappa _{nmt}=1$), GDP per
capita simplifies to:\footnote{%
See derivations in the Appendix.} 
\begin{equation*}
Y_{nt}=\left( \xi B\right) ^{1/\beta }Z_{nt}^{\frac{1}{1+\beta \theta }%
}\left( \sum_{m=1}^{N}Z_{mt}^{\frac{1}{1+\beta \theta }}\right) ^{\frac{1}{%
\beta \theta }} 
\end{equation*}%
and hence GDP fluctuations are given by: 
\begin{equation*}
\hat{Y}_{nt}=\frac{1}{1+\beta \theta }\left[ \hat{Z}_{n}+\frac{1}{\beta
\theta }\sum_{m=1}^{N}\gamma _{m}\hat{Z}_{m}\right] 
\end{equation*}%
where $\gamma _{m}=\frac{\bar{Z}_{m}^{\frac{1}{1+\beta \theta }}}{%
\sum_{i=1}^{N}\bar{Z}_{i}^{\frac{1}{1+\beta \theta }}}$ is the relative size
of country $j$ evaluated at the mean of $Z_{j}s$. Rearranging, we obtain: 
\begin{equation}
\hat{Y}_{nt}=\frac{1}{\beta \theta }\left[ \frac{\gamma _{n}+\beta \theta }{%
1+\beta \theta }\hat{Z}_{n}+\frac{1}{1+\beta \theta }\sum_{m\neq
n}^{N}\gamma _{m}\hat{Z}_{m}\right]
\end{equation}%
Volatility under free trade is hence given by: 
\begin{equation}
Var(\hat{Y}_{nt})=\left( \frac{1}{\beta \theta }\right) ^{2}\left\{ 
\begin{array}{c}
\left( \frac{\gamma _{n}+\beta \theta }{1+\beta \theta }\right) ^{2}Var(\hat{%
Z}_{nt})+\left[ \frac{1}{1+\beta \theta }\right] ^{2}\sum_{m\neq i}\gamma
_{m}^{2}Var(\hat{Z}_{mt}) \\ 
2\frac{\gamma _{n}+\beta \theta }{1+\beta \theta }\frac{1}{1+\beta \theta }%
\sum_{m\neq n}\gamma _{m}Cov(\hat{Z}_{m,}\hat{Z}_{n})%
\end{array}%
\right\}
\end{equation}%
Compared to the variance in autarky, $V(\hat{Y}_{nt})=\frac{1}{\left( \beta
\theta \right) ^{2}}V(\hat{Z}_{nt})$, it is clear that the volatility due to
domestic productivity fluctuations, $Var(\hat{Z}_{nt}),$ now receives a
smaller loading, as $\left( \frac{\gamma _{n}+\beta \theta }{1+\beta \theta }%
\right) ^{2}<1$ since $\gamma _{n}<1.$ The smaller the country (as gauged by
its share $\gamma _{n}$), the smaller the impact of domestic volatility of
shocks, $\hat{Z}_{n},$ on its GDP, when compared to autarky. Openness to
trade, however, exposes the economy to other countries' productivity shocks,
which will also contribute to the country's overall volatility. Whether or
not the gain in diversification (given by lower exposure to domestic
productivity) is bigger than the increased exposure to new shocks depends on
the variance-covariance matrix of shocks across countries. If all countries
have the same constant variance $Var(\hat{Z}_{nt})=\sigma ,$ and the $\hat{Z}%
_{nt}$ are uncorrelated, volatility under free trade becomes: 
\begin{equation}
Var(\hat{Y}_{nt})=\left( \frac{1}{\beta \theta }\right) ^{2}\left\{ \left( 
\frac{\gamma _{n}+\beta \theta }{1+\beta \theta }\right) ^{2}+\left[ \frac{1%
}{1+\beta \theta }\right] ^{2}\sum_{m\neq i}\gamma _{m}^{2}\right\} \sigma
\end{equation}%
which is unambiguously lower than the volatility in autarky given that%
\footnote{%
since $\left( \beta \theta \right) ^{2}+2\beta \theta \gamma
_{n}+\sum_{j=1}\gamma _{j}^{2}<(1+\beta \theta )^{2}$ as 
\begin{equation*}
2\beta \theta \gamma _{n}+\sum_{j=1}\gamma _{j}^{2}<2\beta \theta +1 
\end{equation*}%
} 
\begin{equation}
\left( \frac{\gamma _{n}+\beta \theta }{1+\beta \theta }\right) ^{2}+\left[ 
\frac{1}{1+\beta \theta }\right] ^{2}\sum_{m\neq i}\gamma _{m}^{2}<1
\label{derivelater}
\end{equation}%
(recall $\gamma _{m}<1$ and\ $\sum_{m=1}^{N}\gamma _{m}^{2}\leq 1)$. Of
course, if other countries have higher variances or the covariance terms are
important, then the weights countries receive matter and the resulting
change in volatility cannot be unambiguously signed.

\section{Mapping the Model into Observables}

\label{mapping_into_observables}

In this section, we connect the model to the data and use it to
quantitatively assess the effect of historical changes in trade barriers on
GDP volatility for a diverse sample of 24\ core countries and an aggregate
of the remaining countries\ to which we refer as \textquotedblleft rest of
the world\textquotedblright\ (ROW).

The equilibrium of the model is characterized by equations (\ref{eq1})-(\ref%
{eq7}). We solve the model\ numerically, for which we need to calibrate the
values of the exogenous trading costs $\kappa _{nmt}^{j}$, the productivity
process $Z_{nt}^{j}$, and the parameters $\alpha ^{j}$, $\beta ^{j}$, $%
\theta $, and $\eta $. We consider 24 sectors in the analysis (agriculture,
22 manufacturing sectors, and services). Throughout the study, services are
treated as a nontradable sector (that is, $\kappa _{nmt}^{j}=0$ for all $%
n\neq m$ and $\kappa _{nmt}^{j}=1$ for $n=m$), whereas agriculture and\ all
manufacturing sectors are treated as tradables, with potentially different
trading costs.

We set $\alpha ^{j}$ so as to match the average share of each sector on
total final uses in the OECD Input-Output tables across all countries. The
betas for each sector are calculated as the ratio of value added to total
output. A detailed description of the data and the calculations are
available in the Appendix.

We allow for a relatively broad parametric range for $\theta $, from $\theta
=2$ to $\theta =8,$ consistent with the estimates in the literature (see
Eaton and Kortum, 2003, Donaldson 2015, and Simonovska and Waugh, 2011). We
use $\theta =4$ as the baseline case, and report the results for other
values when discussing the sensitivity of our results. We calibrate the
elasticity of substitution across varieties $\eta =2$, consistent with Broda
and Weinstein (2006). The results are not sensitive to this parametric
choice.

We explain next how we obtain the processes for $\kappa _{nmt}^{j}$ and $%
Z_{it}^{j}$ using data on sectoral bilateral trade flows, value added,
output, and prices. Before we specify the details, a quick intuition on how
these series are backed-out from the model is as follows. We recover trade
costs $\kappa _{nmt}^{j}$ using information on bilateral trade shares and
gross output at the sectoral level. Intuitively, if two countries trade
little with one another in a given sector (relative to the sectoral gross
output of these countries), this will signal high trade costs between the
countries in that sector. Second, we recover productivities relative to a
benchmark country using the market share of each exporter. If a country has
a high export share in a sector, that is a sign of revealed comparative
advantage, meaning a high relative productivity in the sector relative to
the benchmark country. To calibrate the absolute level of productivities, we
use price data for a benchmark country. We explain the procedure in detail
and with formulas in the next section.

\subsection{Implementation}

\paragraph{Kappas}

In order to perform counterfactual experiments we need to back out the
historical realizations of the exogenous processes. Following the idea in
Head and Ries (2011), we assume that sectoral bilateral trading costs are
symmetric, that is: $\kappa _{nmt}^{j}=\kappa _{mnt}^{j}$, and hence
bilateral trade costs at the sectoral level can be backed out from\ the
data. Indeed, inverting the structural model, we obtain: 
\begin{equation}
\frac{d_{nmt}^{j}d_{mnt}^{j}}{d_{mmt}^{j}d_{nnt}^{j}}=\left( \kappa
_{nmt}^{j}\right) ^{2\theta }.  \label{kappa}
\end{equation}%
The left hand side objects can be measured using data on bilateral imports
and gross output at the sectoral level. Specifically, $d_{nmt}^{j}$ is the
value of exports from $m$ to $n$ in sector $j$ at $t$ relative to total
spending by $n$ on sector $j$ at time $t$, where total spending is measured
as gross output plus imports minus exports by that sector and country at
time $t.$ The share $d_{mmt}^{j}$ is obtained as a residual from the
accounting restriction: 
\begin{equation*}
d_{mmt}^{j}=1-\sum\limits_{n\neq m}^{N}d_{mnt}^{j} 
\end{equation*}%
Hence, for a given value of $\theta $, we can obtain the time series of
trading costs by sector and country-pairs $\left\{ \kappa _{nmt}^{j}\right\} 
$.

\paragraph{Productivity in Tradable Sectors}

To back out the productivities, we proceed as follows. First, using the
formula for $d_{nm}^{j}$ in equation (\ref{eq4}), after some algebra, we
obtain: 
\begin{equation}
d_{nm}^{j}=\frac{\left( B^{j}\right) ^{-\theta }\left( \psi _{m}^{j}\right)
^{\beta ^{j}\theta }Z_{m}^{j}\left( \kappa _{nm}^{j}\right) ^{\theta }\left(
y_{m}^{j}\right) ^{-\theta \beta ^{j}}}{P_{m}^{\theta }\Phi _{n}^{j}}\text{, 
}  \label{produc}
\end{equation}%
where $\psi _{m}^{j}\equiv \frac{L_{m}^{j}}{L_{m}}$ and $y_{m}^{j}\equiv 
\frac{L_{m}^{j}w_{m}^{j}}{P_{m}}$. We can exploit this to recover $%
Z_{m}^{j}. $ In particular, inverting (\ref{produc}) we have: 
\begin{equation}
Z_{mt}^{j}={B^{j}}^{\theta }{\xi }^{\theta }d_{nmt}^{j}\left(
y_{m}^{j}\right) ^{\theta \beta ^{j}}\left( \kappa _{nmt}^{j}\right)
^{-\theta }\left( \frac{P_{nt}^{j}}{P_{mt}}\right) ^{-\theta }\left( \psi
_{mt}^{j}\right) ^{-\theta \beta ^{j}}  \label{productivity}
\end{equation}%
To approximate terms on the right hand side we use data on sectoral import
shares $d_{nmt}^{j}$, sectoral value added $y_{m}^{j}$, sectoral shares $%
\psi _{mt}^{j}$, and aggregate prices $P_{nt}\ $along with the calibrated
parameters. (See the Appendix for more details.) The only terms we cannot
back out directly from data are sectoral prices. We thus use the model in
conjunction with the data to infer them. Note first that equation~(\ref%
{productivity}) holds for all $(n,k)$ pairs of countries and all sectors $j$
(except for services). The procedure becomes clear when we collect known and
unknown terms as follows:%
\begin{align}
Z_{nt}^{j}& =\underbrace{\xi ^{\theta }{B^{j}}^{\theta }d_{k,n,t}^{j}\left(
\kappa _{k,n,t}^{j}\right) ^{-\theta }\left( w_{n,t}^{j}L_{n,t}^{j}\right)
^{\theta \beta ^{j}}\left( \psi _{n,t}^{j}\right) ^{-\theta \beta ^{j}}{%
P_{n,t}}^{\theta (1-\beta ^{j})}}_{\equiv \exp (\zeta _{k,n,t}^{j})}{%
P_{k,t}^{j}}^{-\theta }  \notag \\
& =\exp (\zeta _{k,n,t}^{j}){P_{k,t}^{j}}^{-\theta }
\end{align}%
Note in particular that this relationship holds for any choice of country $k$%
. Note also that the factor $\exp (\zeta _{k,n,t}^{j})$ can be constructed
from observable data. We decompose $\exp (\zeta
_{k,n,t}^{j})=Z_{nt}^{j}\left( {P_{k,t}^{j}}\right) ^{\theta }$ according to
the following procedure:

\begin{enumerate}
\item Take logs and rename terms for brevity. 
\begin{align}
\zeta _{k,n,t}^{j}& =\ln {Z_{nt}^{j}}+\theta \ln {P_{k,t}^{j}} \\
& \equiv \chi _{nt}^{j}+\tau _{k,t}^{j}
\end{align}%
where $\chi _{nt}^{j}\equiv \ln {Z_{nt}^{j}}$ and $\tau _{k,t}^{j}\equiv
\theta \ln {P_{k,t}^{j}}$.

\item To proceed we need a benchmark country, so we use sectoral prices in
the US. 
\begin{equation*}
\tau _{US,t}^{j}\equiv \theta \ln {P_{US,t}^{j}} 
\end{equation*}%
We choose units of accounts for each sector so that U.S. nominal sectoral
prices are equal to 1 in 1972.

\item Obtain $\tau _{k,t}^{j}$ for all other countries as: 
\begin{equation}
\tau _{k,t}^{j}=\frac{1}{N}\sum_{n=1}^{N}\left( \zeta _{k,n,t}^{j}-\zeta
_{US,n,t}^{j}\right) +\tau _{US,t}^{j}
\end{equation}%
(Note that this equation holds with and without the averaging operator, $%
\frac{1}{N}\sum_{n=1}^{N}$, as $\tau _{k,t}^{j}$ and $\tau _{US,t}^{j}$ do
not depend on the exporter $n$.\footnote{%
We use the average in the quantitative analysis to minimize measurement
error.})

\item Back-out $\chi _{nt}^{j}$ for all other countries:%
\begin{equation}
\chi _{nt}^{j}=\frac{1}{N}\sum_{k=1}^{N}\left( \zeta _{k,n,t}^{j}-\tau
_{k,t}^{j}\right)
\end{equation}

\item Recover shocks and prices:%
\begin{align}
Z_{n,t}^{j}& =\exp \left( \chi _{nt}^{j}\right) \\
P_{k,t}^{j}& =\exp \left( \frac{\tau _{kt}^{j}}{\theta }\right)
\end{align}
\end{enumerate}

At the end of the procedure we end up with augmented productivity factors $%
Z_{n,t}^{j}$ and sectoral prices for agriculture and all manufacturing
sectors $P_{k,t}^{j}$.

\paragraph{Productivity in Nontradables}

To compute the productivities in the services sector for each country, we
use equilibrium equations (\ref{eq1}), (\ref{eq2}) and (\ref{produc}).

\begin{enumerate}
\item As we already have sectoral prices of tradables we can use (\ref{eq1})
to recover the price of services as follows: 
\begin{equation}
P_{n,t}^{s}=\left( \frac{P_{n,t}}{P_{US,t}}P_{US,t}\right) ^{\frac{1}{\alpha
^{s}}}\left( \prod_{j=1}^{J}{\alpha ^{j}}^{-\alpha ^{j}}\right) ^{-\frac{1}{%
\alpha ^{s}}}\left[ \prod_{j\neq s}\left( P_{n,t}^{j}\right) ^{\alpha ^{j}}%
\right] ^{-\frac{1}{\alpha ^{s}}}
\end{equation}%
Note that we observe data on the price of country $n$ relative to the price
in the United States, $\frac{P_{n,t}}{P_{US,t}}$, from the Penn World Tables.

\item Now we recover $Z_{n,t}^{s}$ using (\ref{eq2}), (\ref{produc}), and $%
n=m$. 
\begin{equation}
Z_{n,t}^{s}=\xi ^{\theta }{B^{s}}^{\theta }\left( \frac{%
w_{n,t}^{s}L_{n,t}^{s}}{\psi _{n,t}^{s}}\right) ^{\theta \beta ^{s}}\left( 
\frac{P_{n,t}}{P_{US,t}}P_{US,t}\right) ^{\theta (1-\beta ^{s})}{P_{n,t}^{s}}%
^{-\theta }
\end{equation}
\end{enumerate}

\paragraph{Sectoral versus Aggregate Shocks}

Note that the changes in productivity retrieved above, 
\begin{equation}
\frac{1}{\beta ^{j}\theta }\hat{Z}_{m}^{j}\equiv \frac{1}{\beta ^{j}}\hat{A}%
_{mt}^{j}+\hat{L}_{mt},  \label{prodchanges}
\end{equation}%
can be decompose into two factors: a sectoral factor, $\frac{1}{\beta ^{j}}%
\hat{A}_{mt}^{j}$, and an aggregate factor $\hat{L}_{mt}$. The
interpretation of $L_{mt}$ as \textquotedblleft equipped
labour\textquotedblright\ means that it embeds a productivity component too.
Given the functional form, the split between pure productivity and resources
in $L_{mt}$ is not relevant from the point of view of aggregate volatility.
(A shock to $L_{mt}$ will be equivalent to an aggregate shock to $A_{mt}^{j}$%
s that leaves the relative productivities $A_{mt}^{j}/A_{mt}^{j^{\prime }}$
unchanged $\forall j,j^{\prime }.$) For identification, we impose the
restriction that 
\begin{equation}
\sum \frac{\alpha ^{j}}{\beta ^{j}}\hat{A}_{mt}^{j}=0.  \label{idrestriction}
\end{equation}%
Thus, changes in the sectoral productivity will correspond to changes in the
relative value of $A_{mt}^{j}$, while changes in aggregate productivity
(affecting all sectors equally), as well as changes in overall resources,
will be subsumed in $L_{mt}$. We hence call sectoral shocks, those affecting 
$\hat{A}_{mt}^{j}$ and aggregate shocks those affecting the aggregate factor 
$\hat{L}_{mt}$. The identification restriction implies that any primitive
aggregate shock affecting all sectors will be collected in $\hat{L}_{mt}.$

\paragraph{Summary of the Procedure}

We can summarize the procedure as follows.

\begin{enumerate}
\item Obtain the\ inverse of trade costs, $\kappa $s, from (\ref{kappa}).

\item Compute $\psi _{mt}^{j}$ as the sectoral value-added share at time $t$.

\item Retrieve the panel of sectoral and country productivities$\left\{
Z_{mt}^{j}\right\} $ from the procedure described above.\footnote{%
Units of accounts are chosen so that nominal sectoral prices in the US in
1972 equal 1.}

\item Retrieve $L_{mt}$ from (\ref{prodchanges}) using (\ref{idrestriction})
and compute $L_{mt}^{j}=\psi _{mt}^{j}L_{mt}.$

\item Solve the equilibrium values of $\left\{ d_{nit}^{j}\right\} $, $%
\left\{ P_{nt}\right\} $, $\left\{ P_{nt}^{j}\right\} $, and $\left\{
w_{n}^{j}\right\} $ using equations (\ref{eq1}) through (\ref{eq6}).
\end{enumerate}

\subsubsection{Counterfactual Equilibria}

We discuss next how we compute the equilibrium in the counterfactual
exercise and how we identify the two theoretical mechanisms.

\paragraph{Numerical Counterfactual Equilibria}

For each new value of (inverse) trading cost $\kappa $, and the estimated
sequence of sectoral productivities $\left\{ Z_{mt}^{j}\right\} $, we need
to solve for the sequence of equipped labour allocated to each sector $%
\left\{ L_{mt}^{j}\right\} $. The rational-expectations equilibrium is a
fixed point of a below mapping on the space of all possible sequences $%
\left\{ L_{mt}^{j}\right\} $. We proceed as follows.

\begin{enumerate}
\item We start from the initial value $(L_{nt}^j)^0 = \alpha^j L_{nt}$.

\item In iteration $i$ for the actual $(L_{nt}^{j})^{i}$ we get sectoral and
aggregate (equipped labour) wages, $(w_{nt}^{j})^{i}$ and $(w_{nt})^{i}$,
from the equilibrium equations.

\item We calculate the implied total value added and the sectoral value
added shares as 
\begin{equation*}
\left(\frac{w_{nt}^j L_{nt}^j}{w_{nt} L_{nt}}\right)^i = \frac{(w_{nt}^j)^i
(L_{nt}^j)^i}{(w_{nt})^i L_{nt}}. 
\end{equation*}

\item 
\begin{enumerate}
\item Decompose all $N\cdot J$ value-added-share series into trend and cycle
components using an annual band-pass filter. 
\begin{equation*}
\log \left( \frac{w_{nt}^{j}L_{nt}^{j}}{w_{nt}L_{nt}}\right)
^{i}=trend_{nt}^{j}+cycle_{nt}^{j}. 
\end{equation*}

\item Normalize the trend values so that in each period and each country the
trend values add up to 1: 
\begin{equation*}
\widehat{\left( \frac{w_{nt}^{j}L_{nt}^{j}}{w_{nt}L_{nt}}\right) ^{i}}=\frac{%
\exp (trend_{nt}^{j})}{\sum_{k}\exp (trend_{nt}^{k})} 
\end{equation*}

\item Replace the expectation with the adjusted trend value. 
\begin{equation*}
E_{t - 1} \left( \frac{w_{nt}^j L_{nt}^j}{w_{nt} L_{nt}}\right) = \widehat{%
\left(\frac{w_{nt}^j L_{nt}^j}{w_{nt} L_{nt}}\right)^i} 
\end{equation*}
\end{enumerate}

\item Update the resource allocations 
\begin{equation*}
(L_{nt}^j)^{i + 1} = L_{nt} \widehat{\left(\frac{w_{nt}^j L_{nt}^j}{w_{nt}
L_{nt}}\right)^i} 
\end{equation*}

\item Repeat the procedure until convergence.
\end{enumerate}

\paragraph{Productivities in Counterfactual Scenario}

We are interested in decomposing the trade effect on volatility on the
contributions of the two mechanisms, specialization and diversification. To
achieve that, we need to identify the sources of shocks to productivity. We
resort to a factor model that decomposes productivity shocks into sector-
and country-specific components in a way described in Koren and Tenreyro
(2007). To separate per period shocks from trends we use a band pass filter
to detrend each $\left\{ \log {Z_{n,t}^{j}}\right\} _{t=1}^{T}$ series. Then
we calculate the time average of the shocks for each $(n,j)$ pair and
subtract it from the growth rate to get the object to be decomposed, $\tilde{%
Z}_{nt}^{j}$. 
\begin{equation*}
\tilde{Z}_{nt}^{j}=\hat{Z}_{n,t}^{j}-(T-1)^{-1}\sum_{t=2}^{T}\hat{Z}%
_{n,t}^{j} 
\end{equation*}%
Without loss of generality, we decompose $\tilde{Z}_{nt}^{j}$ as: 
\begin{equation*}
\tilde{Z}_{nt}^{j}=\lambda _{t}^{j}+\mu _{nt}+\epsilon _{nt}^{j}, 
\end{equation*}%
where ${\mu _{n,t}}$ is the country-specific factor, affecting all sectors
within the country; $\lambda _{t}^{j}$ is the global sectoral factor,
affecting sector $j$ in all countries;\ and the residual $\epsilon
_{n,t}^{j} $ is the idiosyncratic component, specific to the country and
sector. The three factors, $\lambda ,\mu $, and $\epsilon $ are estimated
as: 
\begin{align*}
\hat{\lambda}_{t}^{j}& =N^{-1}\sum_{n=1}^{N}\tilde{Z}_{nt}^{j} \\
\hat{\mu}_{nt}& =J^{-1}\sum_{j=1}^{J}\left( \tilde{Z}_{nt}^{j}-\hat{\lambda}%
_{t}^{j}\right) \\
\hat{\epsilon}_{nt}^{j}& =\tilde{Z}_{nt}^{j}-\hat{\lambda}_{t}^{j}-\hat{\mu}%
_{nt}\text{,}
\end{align*}%
with the restriction $\sum_{n}{\mu _{n}}=0$ implying that the
country-specific effect is expressed relative to the world's aggregate. In
the counterfactual exercises, we can mute the sector- or country-specific
factors by setting the corresponding components equal to 0, in order to
identify the separate effects of the two trade channels affecting volatility.

\end{document}
