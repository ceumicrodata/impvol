%2multibyte Version: 5.50.0.2953 CodePage: 65001

\documentclass{article}
%%%%%%%%%%%%%%%%%%%%%%%%%%%%%%%%%%%%%%%%%%%%%%%%%%%%%%%%%%%%%%%%%%%%%%%%%%%%%%%%%%%%%%%%%%%%%%%%%%%%%%%%%%%%%%%%%%%%%%%%%%%%%%%%%%%%%%%%%%%%%%%%%%%%%%%%%%%%%%%%%%%%%%%%%%%%%%%%%%%%%%%%%%%%%%%%%%%%%%%%%%%%%%%%%%%%%%%%%%%%%%%%%%%%%%%%%%%%%%%%%%%%%%%%%%%%
%TCIDATA{OutputFilter=LATEX.DLL}
%TCIDATA{Version=5.50.0.2953}
%TCIDATA{Codepage=65001}
%TCIDATA{<META NAME="SaveForMode" CONTENT="1">}
%TCIDATA{BibliographyScheme=Manual}
%TCIDATA{Created=Thursday, January 19, 2017 14:01:11}
%TCIDATA{LastRevised=Friday, January 20, 2017 11:13:43}
%TCIDATA{<META NAME="GraphicsSave" CONTENT="32">}
%TCIDATA{<META NAME="DocumentShell" CONTENT="Standard LaTeX\Blank - Standard LaTeX Article">}
%TCIDATA{CSTFile=40 LaTeX article.cst}

\newtheorem{theorem}{Theorem}
\newtheorem{acknowledgement}[theorem]{Acknowledgement}
\newtheorem{algorithm}[theorem]{Algorithm}
\newtheorem{axiom}[theorem]{Axiom}
\newtheorem{case}[theorem]{Case}
\newtheorem{claim}[theorem]{Claim}
\newtheorem{conclusion}[theorem]{Conclusion}
\newtheorem{condition}[theorem]{Condition}
\newtheorem{conjecture}[theorem]{Conjecture}
\newtheorem{corollary}[theorem]{Corollary}
\newtheorem{criterion}[theorem]{Criterion}
\newtheorem{definition}[theorem]{Definition}
\newtheorem{example}[theorem]{Example}
\newtheorem{exercise}[theorem]{Exercise}
\newtheorem{lemma}[theorem]{Lemma}
\newtheorem{notation}[theorem]{Notation}
\newtheorem{problem}[theorem]{Problem}
\newtheorem{proposition}[theorem]{Proposition}
\newtheorem{remark}[theorem]{Remark}
\newtheorem{solution}[theorem]{Solution}
\newtheorem{summary}[theorem]{Summary}
\newenvironment{proof}[1][Proof]{\noindent\textbf{#1.} }{\ \rule{0.5em}{0.5em}}
\input{tcilatex}

\begin{document}


The world economy is composed of $N$ countries. At a given point in time $t$%
, each country $n$ is endowed with $L_{nt}$ units of a primary (non
produced) input, which we interpret as equipped labour. There are $J$
sectors (or broad classes of goods) in the economy, whose output is combined
into a final good through a Cobb-Douglas aggregate. In formulas, aggregate
gross output in the economy is given by: 
\begin{equation}
C_{nt}=\left[ \sum\nolimits_{j=1}^{J}\alpha ^{j}\left( C_{nt}^{j}\right) ^{%
\frac{\sigma -1}{\sigma }}\right] ^{\frac{\sigma }{\sigma -1}}
\label{aggregate}
\end{equation}%
where $C_{t}^{j}$ is the aggregate composite in sector $j$ and $%
\sum_{j=1}^{J}\alpha ^{j}=1$. Competitive firms in each sector $j$ produce a
composite good according to the following
constant-elasticity-of-substitution (CES) technology: 

\bigskip min $P_{nt}^{j}C_{nt}^{j}$ st: $\left[ \sum\nolimits_{j=1}^{J}%
\alpha ^{j}\left( C_{nt}^{j}\right) ^{\frac{\sigma -1}{\sigma }}\right] ^{%
\frac{\sigma }{\sigma -1}}=1$

\bigskip FOC%
\begin{eqnarray*}
P_{nt}^{j} &=&\lambda \left[ \sum\nolimits_{j=1}^{J}\alpha ^{j}\left(
C_{nt}^{j}\right) ^{\frac{\sigma -1}{\sigma }}\right] ^{\frac{\sigma }{%
\sigma -1}-1}\alpha ^{j}\left( C_{nt}^{j}\right) ^{\frac{-1}{\sigma }} \\
P_{nt}^{j} &=&\lambda \alpha ^{j}\left( C_{nt}^{j}\right) ^{\frac{-1}{\sigma 
}} \\
\left( P_{nt}^{j}\right) ^{1-\sigma } &=&\lambda ^{1-\sigma }\left( \alpha
^{j}\right) ^{1-\sigma }\left( C_{nt}^{j}\right) ^{\frac{\sigma -1}{\sigma }}
\\
\left( \alpha ^{j}\right) ^{\sigma }\left( P_{nt}^{j}\right) ^{1-\sigma }
&=&\lambda ^{1-\sigma }\left( \alpha ^{j}\right) \left( C_{nt}^{j}\right) ^{%
\frac{\sigma -1}{\sigma }} \\
\sum\nolimits_{j=1}^{J}\left( \alpha ^{j}\right) ^{\sigma }\left(
P_{nt}^{j}\right) ^{1-\sigma } &=&\lambda ^{1-\sigma
}\sum\nolimits_{j=1}^{J}\left( \alpha ^{j}\right) \left( C_{nt}^{j}\right) ^{%
\frac{\sigma -1}{\sigma }} \\
\left[ \sum\nolimits_{j=1}^{J}\left( \alpha ^{j}\right) ^{\sigma }\left(
P_{nt}^{j}\right) ^{1-\sigma }\right] ^{^{\frac{\sigma }{\sigma -1}}}
&=&\lambda ^{-\sigma }\left[ \sum\nolimits_{j=1}^{J}\left( \alpha
^{j}\right) \left( C_{nt}^{j}\right) ^{\frac{\sigma -1}{\sigma }}\right] ^{%
\frac{\sigma }{\sigma -1}} \\
\left[ \sum\nolimits_{j=1}^{J}\left( \alpha ^{j}\right) ^{\sigma }\left(
P_{nt}^{j}\right) ^{1-\sigma }\right] ^{^{\frac{1}{1-\sigma }}} &=&\lambda 
\end{eqnarray*}

\bigskip 

\bigskip 

\bigskip 

\begin{equation}
Q_{nt}^{j}=\left[ \int_{0}^{1}q_{nt}(\omega ^{j})^{\frac{\eta -1}{\eta }%
}d\omega ^{j}\right] ^{\frac{\eta }{\eta -1}}  \label{sectoroutput}
\end{equation}%
where $q_{nt}(\omega ^{j})$ is the quantity of good $\omega ^{j}$ used by
country $n$ in sector $j$ at time $t$, and $\eta >0$ is the elasticity of
substitution across goods within a given sector. The intermediate goods $%
\omega ^{j}$ can be produced locally or imported from other countries.
Delivering a good from country $n$ to country $m$ in sector $j$ and time
period $t$ results in $0<\kappa _{mnt}^{j}\leq 1$ goods arriving at $m$; we
assume that $\kappa _{mnt}^{j}\geq \kappa _{mkt}^{j}\kappa _{knt}^{j}\quad
\forall m,n,k,j,t$ and $\kappa _{nnt}^{j}=1$. All costs incurred are net
losses.\footnote{%
In the calibration, the $\kappa $s will reflect all trading costs, including
tariffs; so implicitly we adopt the extreme assumption that tariff revenues
are wasted---or at least not rebated back to agents in a way that would
interact with the allocation of resources in the economy.} Under the
assumption of perfect competition, goods are sourced from the lowest-cost
producer, after adjusting for transport costs.

The technology for producing good $\omega ^{j}$ in country $m$ is: 
\begin{equation}
x_{mt}(\omega ^{j})=A_{mt}^{j}z_{m}(\omega ^{j})l_{mt}(\omega ^{j})^{\beta
^{j}}\prod\nolimits_{k=1}^{J}M_{mt}^{k}(\omega ^{j})^{\gamma ^{kj}}
\label{eqinput}
\end{equation}%
where $x_{mt}(\omega ^{j})$ is the production of good $\omega ^{j}$ by
country $m$ at time $t$; $M_{mt}^{k}(\omega ^{j})$ is the amount of sector $%
k $ output used by country $m$ to produce $x_{mt}(\omega ^{j})$ units of
good $\omega ^{j};$ $l_{mt}(\omega ^{j})$ is the corresponding amount of
equipped labour; $z_{n}(\omega ^{j})$ is a time-invariant variety-specific
productivity factor; and $A_{nt}^{j}$ is a time-varying productivity shock
common to all the varieties in sector $j$. The exponent $\gamma ^{kj}$
captures the share of sector $k$ in the total production cost of sector $j.$
We assume constant returns to scale, or $\beta ^{j}+\sum_{k=1}^{J}\gamma
^{kj}=1$, for all $j$. As such, the expression allows for a rich
input-output structure.

Building on the literature, we assume the productivities $z_{n}(\omega ^{j})$
follow a sector-specific, time-invariant Fr\'{e}chet distribution $%
F_{n}^{j}(z)=\exp (-T_{n}^{j}z^{-\theta })$. A higher $T_{n}^{j}$ shifts the
distribution of productivities to the right, that is leading to
probabilistically higher productivities. A higher $\theta $ decreases the
dispersion of the productivity distribution, and hence reduces the scope for
comparative advantage. Shocks to $A_{nt}^{j}$ over time are interpreted as
standard sectoral TFP shocks. The $z$\ terms are the main determinants of
long-term comparative advantage in our model.

Only intermediate goods $\omega ^{j}$ are traded; final composites $%
Q_{nt}^{j}$ are not. These final goods are used directly by consumers and
also as inputs in the production process. Hence, market clearing in the good
markets implies: 
\[
Q_{nt}^{j}=C_{nt}^{j}+\sum\nolimits_{k=1}^{J}\int_{0}^{1}M_{nt}^{j}(\omega
^{k})d\omega ^{k},
\]%
where the integral aggregates over the unit-size continuum of goods $\omega
^{k}$ entering in the production of each sector's $k$ aggregate good.

Clearing in the input market within a sector implies: 
\[
L_{nt}^{j}=\int_{0}^{1}l_{nt}(\omega ^{j})d\omega ^{j},
\]%
where $l_{nt}(\omega ^{j})$ denotes the amount of equipped labour used in
the production of good $\omega ^{j}$ by country $n$. The (equipped) labour
allocated to each sector, $L_{nt}^{j},$ with $\sum%
\nolimits_{k=1}^{J}L_{nt}^{k}=L_{nt}$, is determined ex ante (before the
realization of the shocks). Specifically, we assume there is perfect
risk-sharing within a country, but no risk-sharing across countries.%
\footnote{%
To motivate the lack of risk-sharing across countries, see our earlier
discussion on the high comovement between consumption and output as well as
the high correlation between consumption and output volatility.} At the
beginning of each period, a representative consumer decides on the optimal
allocation of the primary input $L_{nt}$ into different sectors in order to
maximize the expected value of utility; then (stochastic) shocks to
productivity $A_{nt}^{j}$ are realized, equipped labour is reallocated
within a sector (but not across sectors), and production and consumption
take place. The lack of ex-post reallocation across sectors in a given
period aims at capturing the idea that in the short run, it is costly to
reallocate productive factors across sectors. Hence, ex post, $L_{t}^{j}$ is
fixed until $t+1.$\footnote{%
In the quantification, a period will be one year. This amounts to assuming
that it takes at least one year for resources to be reallocated across
sectors.} We relax this assumption in Section V, where we allow for ex post
sectoral reallocation of equipped labour. (The reallocation carries a
quadratic adjustment cost, which we calibrate to match sectoral reallocation
in the data.)

The representative consumer's budget constraint in each period is: 
\begin{equation}
P_{nt}C_{nt}=\sum\nolimits_{j=1}^{J}w_{nt}^{j}L_{nt}^{j}-S_{nt},  \label{BC}
\end{equation}%
where $P_{nt}$ is the price of the consumption good defined in equation (\ref%
{aggregate}), $w_{nt}^{j}L_{nt}^{j}$ is the nominal value-added generated in
sector $j$, and $S_{nt}$ is an exogenously given current account surplus of
country $n$, with $\sum_{n}S_{nt}=0$ for all $t$. We treat the
current-account surplus as exogenous because the computational difficulties
of endogenizing it are formidable.

Lifetime utility is given by 
\[
U_{n}=\sum\nolimits_{t=0}^{\infty }\delta ^{t}u(C_{nt}),
\]%
where $u^{\prime }>0$, $u^{\prime \prime }\leq 0$ and $\delta $ is the
discount factor. Because there is no intertemporal trade and no capital in
the economy, each period consumers maximize the expected static utility flow 
$E\left[ u\left( C_{nt}\right) \right] $ and the equilibrium is simply a
sequence of static equilibria. In making his labor allocation decisions the
representative consumer takes into account the joint probability
distribution function of sectoral productivities, $A_{nt}^{j}$s. In the
analysis, we assume log utility and therefore the consumer solves: 
\begin{equation}
L_{nt}^{j}=\arg \max E_{t-1}\left[ \ln \left( \frac{%
\sum_{j=1}^{J}w_{nt}^{j}L_{nt}^{j}}{P_{nt}}\right) \right]
,s.t.:\sum\nolimits_{j=1}^{J}L_{nt}^{j}=L_{nt},  \label{eq:log:utility}
\end{equation}%
where $E_{t-1}$ indicates that the expectation is taken before the
realization of period $t$ shocks.\textbf{\ }

\textbf{Shouldn't there be a }$-S_{nt}$ \textbf{in the bracket?}

\subsection{Model Solution}

Conditional on the realization of the country-and-sector specific shocks $%
A_{nt}^{j}$, our model is very similar to other general equilibrium,
multi-sector versions of the Eaton-Kortum model. The main difference is that
equipped labor is pre-allocated across sectors. Hence, we do not offer a
detailed derivation of the key equilibrium conditions that are unaffected by
the ex-ante allocation of resources, but merely state them in the following
list.

\begin{equation}
d_{nmt}^{j}=\frac{T_{m}^{j}\left( \frac{B^{j}\left( w_{mt}^{j}\right)
^{\beta ^{j}}\prod_{k=1}^{J}(P_{mt}^{k})^{\gamma ^{kj}}}{A_{mt}^{j}\kappa
_{nmt}^{j}}\right) ^{-\theta }}{\sum_{m=1}^{N}T_{m}^{j}\left( \frac{%
B^{j}\left( w_{mt}^{j}\right) ^{\beta
^{j}}\prod_{k=1}^{J}(P_{mt}^{k})^{\gamma ^{kj}}}{A_{mt}^{j}\kappa _{nmt}^{j}}%
\right) ^{-\theta }},  \label{shares}
\end{equation}

\begin{equation}
P_{nt}^{j}=\xi \sum_{m=1}^{N}T_{m}^{j}\left( \frac{B^{j}\left(
w_{mt}^{j}\right) ^{\beta ^{j}}\prod_{k=1}^{J}(P_{mt}^{k})^{\gamma ^{kj}}}{%
A_{mt}^{j}\kappa _{nmt}^{j}}\right) ,  \label{eq2}
\end{equation}

\begin{equation}
P_{nt}=\prod\nolimits_{j}^{J}\left( \frac{1}{\alpha _{n}^{j}}\right)
^{\alpha ^{j}}\left( P_{nt}^{j}\right) ^{\alpha ^{j}},  \label{eq1}
\end{equation}%
\begin{equation}
R_{nt}^{j}=\sum\nolimits_{m=1}^{N}d_{mnt}^{j}E_{mt}^{j},  \label{rev}
\end{equation}%
\begin{equation}
E_{mt}^{j}=\alpha _{t}^{j}P_{mt}C_{mt}+\sum\nolimits_{k=1}^{J}\gamma
^{jk}R_{mt}^{k},  \label{exp}
\end{equation}%
\begin{equation}
w_{mt}^{j}L_{nt}^{j}=\beta ^{j}R_{mt}^{j},  \label{RC}
\end{equation}%
and the budget constraint (\ref{BC}). In the equations above, $d_{nmt}^{j}$
is the fraction of country $n$'s total spending on sector-$j$ goods that is
imported from country $m;$ $P_{nt}^{j}$ is the price of sectoral-good $j$ in
country $n$; $R_{n}^{j}$ is total revenues accruing to firms operating in
sector $j$ in country $n$; and $E_{n}^{j}$ is total expenditure by country $%
n $ residents (consumers and firms) on sectoral good $j$. $B^{j}\equiv
\left( \beta ^{j}\right) ^{-\beta ^{j}}\dprod\limits_{k=1}^{J}\left( \gamma
^{k_{j}}\right) ^{-\gamma ^{k_{j}}}$ and $\xi \equiv \left[ \Gamma \left( 
\frac{\theta +1-\eta }{\theta }\right) \right] $, where $\Gamma $ is the
gamma function, are parametric constants. Hence, equation (\ref{shares})
says that country $n$ imports disproportionately from countries $m$ and
sectors $j$ that have high productivity draws $T_{m}^{j}$ and $A_{m}^{j}$;
low wages $w_{m}^{j}$ and sectoral prices $P_{m}^{k}$; and low bilateral
trading costs, namely high $\kappa _{nm}$s. Equation (\ref{eq2}) says that
the same factors affect domestic sectoral prices. Equation (\ref{eq1})
follows from the final-good producer's profit maximization problem, and
shows the price of consumption as an aggregate of the sectoral prices.
Equation (\ref{rev}) expresses the total sales of sector $j$ in country $n$
as a function of each country's expenditures on that sector and the share of
country $n$ in each country's imports in that sector. Equation (\ref{exp})
states that a country's expenditures in sector $j$ is the sum of final and
intermediate uses of sector $j$ goods. Equation (\ref{RC}) simply notes from
the Cobb-Douglas formulation that value added from sector $j$ is a share $%
\beta ^{j}$ of the gross output of sector $j$.

To these fairly standard equilibrium conditions we add here the first-order
conditions for the allocation of inputs to sectors, which turns out to be: 
\begin{equation}
\frac{L_{nt}^{j}}{L_{nt}}=E_{t-1}\left[ \frac{w_{nt}^{j}L_{nt}^{j}}{%
\sum_{k}w_{nt}^{k}L_{nt}^{k}}\right] ,\qquad \forall j,t.  \label{eq:FOC}
\end{equation}%
The share of resources allocated to a given sector equals its expected share
in value added. Note that $1/\sum_{k}w_{nt}^{k}L_{nt}^{k}$ is the marginal
utility of consumption in period $t$; thus, more resources are allocated to
higher value-added sectors, after appropriately weighting by marginal
utility.\footnote{%
Compared to the allocation in a determinisitc model, in our stochastic
application sectors whose productivity is negatively correlated with
aggregate productivity (that is, they have high value added when the rest of
the economy has low value added) are allocated a disproportionate share of
resources. In states of the world in which overall income is low, the
marginal utility of consumption $1/\sum_{k}w_{nt}^{k}L_{nt}^{k}$ will be
high and hence the optimal allocation entails allocating more resources to
this sector.}

The model can conceptually be solved backwards in two steps. First, for any
given set of values for $L_{nt}^{j}$, equations (\ref{shares})-(\ref{RC})
can be solved for $P_{nt}$, $w_{nt}^{j}$, $P_{nt}^{j}$, $d_{nmt}^{j}$, $%
E_{n}^{j}$, $R_{n}^{j}$, and $C_{n}$ as functions of the $\kappa _{mnt}^{j}$%
s, the $T_{n}^{j}$s, the $A_{n}^{j}$s, and of course the $L_{n}^{j}$s. For
calibration purposes it turns out to be both possible and convenient to
express the dependence of these solutions on $T_{n}^{j}$, $A_{n}^{j}$, and $%
L_{n}^{j}$ in terms of the \textit{augmented productivity factors}%
\begin{equation}
Z_{nt}^{j}\equiv T_{n}^{j}\left[ L_{nt}\left( A_{nt}^{j}\right) ^{1/\beta
^{j}}\right] ^{\beta ^{j}\theta },  \label{productivityfactor}
\end{equation}%
and the sectoral employment shares $\frac{L_{nt}^{j}}{L_{nt}}$. The
augmented productivity factors capture the joint influence of all the
exogenous processes (whether deterministic or stochastic) that impinge on
the country and sector overall productive capacity.

The second stage of the solution uses (\ref{eq:FOC}).to find the ex-ante
shares $L_{nt}^{j}/L_{n}$. In using this equation we interpret the
expectation operator as a rational expectation. In principle, therefore, we
endow the representative agent making the labor allocation decisions with
the same knowledge of the exogenous process $\kappa _{nmt}^{j}$ and (the
stochastic properties of) $Z_{nt}^{j}$ as established below in Section \ref%
{mapping_into_observables}. In practice, to solve the model we identify the
expectation in (\ref{eq:FOC}) with the trend value of the expression in
brackets. Hence, our solution method begins with a choice of a candidate
time series for $L_{nt}^{j}$, and computes the solution for $w_{nt}^{j}$
given that candidate time series as well as realizations of the exogenous
processes. It them computes the trend of the expression in brackets, and if
this is (close enough to) equal to $L_{nt}^{j}/L_{nt}$ is stops. Otherwise
the next iteration sets $L_{nt}^{j}/L_{nt}$ equal to the trend value
resulting from the previous iteration. A more detailed explanation is
provided in the Appendix.

The key theoretical outcome we are interested in is aggregate income
volatility, which we measure as the variance (or standard deviation, where
indicated), of real income deviations from country-specific trends. I turn,
real income in the model is given by total value added deflated by the
optimal expenditure-based price index, or $Y_{nt}=\frac{w_{nt}L_{nt}}{P_{nt}}%
.$ As discussed in the Introduction, these welfare-relevant measures of
income are expected to show first-order responses to changes in the terms of
trade, and hence in foreign productivities, endowments, or trade costs.

\subsection{Two Illustrativ}

\end{document}
