%2multibyte Version: 5.50.0.2960 CodePage: 65001

\documentclass[12pt]{article}
%%%%%%%%%%%%%%%%%%%%%%%%%%%%%%%%%%%%%%%%%%%%%%%%%%%%%%%%%%%%%%%%%%%%%%%%%%%%%%%%%%%%%%%%%%%%%%%%%%%%%%%%%%%%%%%%%%%%%%%%%%%%%%%%%%%%%%%%%%%%%%%%%%%%%%%%%%%%%%%%%%%%%%%%%%%%%%%%%%%%%%%%%%%%%%%%%%%%%%%%%%%%%%%%%%%%%%%%%%%%%%%%%%%%%%%%%%%%%%%%%%%%%%%%%%%%
\usepackage{amsmath}

\setcounter{MaxMatrixCols}{10}
%TCIDATA{OutputFilter=LATEX.DLL}
%TCIDATA{Version=5.50.0.2960}
%TCIDATA{Codepage=65001}
%TCIDATA{<META NAME="SaveForMode" CONTENT="1">}
%TCIDATA{BibliographyScheme=Manual}
%TCIDATA{Created=Monday, February 29, 2016 10:30:29}
%TCIDATA{LastRevised=Wednesday, March 16, 2016 08:53:36}
%TCIDATA{<META NAME="GraphicsSave" CONTENT="32">}
%TCIDATA{<META NAME="DocumentShell" CONTENT="Standard LaTeX\Blank - Standard LaTeX Article">}
%TCIDATA{CSTFile=40 LaTeX article.cst}

\newtheorem{theorem}{Theorem}
\newtheorem{acknowledgement}[theorem]{Acknowledgement}
\newtheorem{algorithm}[theorem]{Algorithm}
\newtheorem{axiom}[theorem]{Axiom}
\newtheorem{case}[theorem]{Case}
\newtheorem{claim}[theorem]{Claim}
\newtheorem{conclusion}[theorem]{Conclusion}
\newtheorem{condition}[theorem]{Condition}
\newtheorem{conjecture}[theorem]{Conjecture}
\newtheorem{corollary}[theorem]{Corollary}
\newtheorem{criterion}[theorem]{Criterion}
\newtheorem{definition}[theorem]{Definition}
\newtheorem{example}[theorem]{Example}
\newtheorem{exercise}[theorem]{Exercise}
\newtheorem{lemma}[theorem]{Lemma}
\newtheorem{notation}[theorem]{Notation}
\newtheorem{problem}[theorem]{Problem}
\newtheorem{proposition}[theorem]{Proposition}
\newtheorem{remark}[theorem]{Remark}
\newtheorem{solution}[theorem]{Solution}
\newtheorem{summary}[theorem]{Summary}
\newenvironment{proof}[1][Proof]{\noindent\textbf{#1.} }{\ \rule{0.5em}{0.5em}}
\input{tcilatex}
\begin{document}


\subsection{The Kehoe--Ruhl Critique}

Many scholars, including Kehoe and Ruhl (2008), have documented that
terms-of-trade improvements positively impact real GDP. This empirical
result is consistent with the intuitive and fairly old result that
international trade can generate real productivity gains, mediated by
changes in relative prices.

Kehoe and Ruhl (2008), however, argue that the empirical evidence, far from
intuitive, poses a challenge to standard models of international trade.
Specifically, they argue that if GDP is measured as a chain-weighted
quantity index of value added, then terms-of-trade changes (and in
particular, import price changes) should not affect real GDP up to a
first-order approximation. In what follows, we explain their reasoning---and
the solution to their puzzle. In particular, we explain why, contrary to
Kehoe and Ruhl's theoretical result, terms-of-trade changes lead to changes
in measured real GDP, both in practice and in a broad set of models.

Kehoe and Ruhl (2008) focus on a representative firm producing one final
good using a single\ imported material input, denoted by $m_{t}$, and
labour, $l$, the latter fixed over time (or at least not responding to the
material input's price). Production is characterized as%
\begin{equation}
q_{t}=f(l,m_{t})  \label{quant}
\end{equation}%
where $q_{t}$ is the quantity of the final good sold and $f$ is a concave,
continuously differentiable function with constant returns to scale. Nominal
value added is given by: 
\begin{equation*}
y_{t}=p_{t}^{y}q_{t}-p_{t}^{m}m_{t}\text{,}
\end{equation*}%
where $p_{t}^{y}$ is the price of output and $p_{t}^{m}$ is the price of the
imported product. Real GDP in Kehoe and Ruhl (2008), is constructed as
output minus the cost of inputs, both evaluated at base year prices,
indicated here by $0$: 
\begin{equation}
y_{t}^{\func{real}}=p_{0}^{y}q_{t}-p_{0}^{m}m_{t}.  \label{realgdp}
\end{equation}%
Chain-weighed real GDP growth in Kehoe Ruhl (2008) is defined as:\footnote{%
This formula is another simplification by Kehoe and Ruhl (2008). As
discussed later, the chain-weighted indexes produced by the BLS are actually
based on the Fisher formula, which is a combination of a Paasche and a
Laspeyres index.}%
\begin{equation}
\frac{p_{0}^{y}f(l,m_{t})-p_{0}^{m}m_{t}}{p_{0}^{y}f(l,m_{0})-p_{0}^{m}m_{0}}%
.  \label{chain}
\end{equation}%
As the authors argued, $m_{t}$ is chosen in response to $p_{t}$, so that 
\begin{equation*}
f_{m}(l,m_{t})=\frac{p_{t}^{m}}{p_{t}^{y}}.
\end{equation*}%
Taking a first-order Taylor approximation of $q_{t}=f(l,m_{t})$ around $%
m_{0} $, 
\begin{equation}
q_{t}=f(l,m_{t})\approx f(l,m_{0})+f_{m}(l,m_{0})(m_{t}-m_{0})=f(l,m_{0})+%
\frac{p_{0}^{m}}{p_{0}^{y}}(m_{t}-m_{0})  \label{result}
\end{equation}%
Substituting $q_{t}$ from (\ref{result}) into (\ref{realgdp}), period-$t$
real GDP (to a first-order approximation) is given by: 
\begin{equation*}
y_{t}^{\func{real}}\approx p_{0}^{y}f(l,m_{0})-p_{0}^{m}m_{0},
\end{equation*}%
which is \emph{base-period's} quantities evaluated at base-period's prices.
Hence, the authors conclude that there is no first-order change in the 
\textit{measured} chain-weighted quantity index, as can be easily seen by
replacing the latest expression into (\ref{chain}). In other words, any
change in $p_{t}^{m}$ from the base-period price $p_{0}^{m}$ will not affect
measured real GDP.

\subsubsection{Multiple goods}

Kehoe and Ruhl (2008)'s conclusion is correct in the one-good economy they
analyse, which cannot capture the nuisances of price index contruction in a
multiple-good setting. We explain next how in a setting with multiple goods,
their argument breaks down, given the way price indexes are computed in
practice. We focus on a very simple example to keep the parallel with Kehoe
and Ruhl (2008), but the argument goes through in a more complex setting.

Suppose that the input bundle consists of multiple goods indexed by $i$.
Different inputs $i$ are imperfect substitutes, but any single $i$ can be
sourced from either the home country, denoted by $H,$ or the foreign
country, $F$.\footnote{%
To illustrate the idea, consider an input bundle of the form $m_{t}=\left\{
\int_{0}^{1}[m_{Ft}(i)+m_{Ht}(i)]^{\alpha }di\right\} ^{1/\alpha },$ where
different inputs $i$ have a constant elasticity of substituion, and each
input $i$ can be supplied by a foreign or domestic producer.(This particular
functional form is not needed to make our point, which is more general.)
\par
In practice (as in our model), there will be multiple foreign countries
potentially supplying the good. This creates even more margins of supplier
substitution, and will reinfornce the result that changes in prices of
foreign inputs lead to changes in measured real GDP.} In equilibrium, good $i
$ is supplied by the lowest-cost producer. We assume as Kehoe-Ruhl, that
there is a single final good whose price is denoted by $p_{t}^{y}$ and whose
quantity produced is represented by $q_{t}$. Denoting by $p_{Ht}(i)$ the
domestic cost of input $i$ at time $t$; $p_{Ft}(i)$ the cost of good $i$
when bought from the foreign producer; and $m_{Ht}(i)$ and $m_{Ft}(i)$ the
corresponding quantities of the inputs, nominal GDP is given by: 
\begin{equation}
y_{t}=p_{t}^{y}q_{t}-\sum\limits_{i\in \mathcal{\Psi }%
_{t}}p_{Ht}(i)m_{Ht}(i)-\sum\limits_{i\in \mathcal{M}_{t}}p_{Ft}(i)m_{Ft}(i)
\label{nominal}
\end{equation}%
where $\mathcal{M}_{t}$ denotes the set of products in which the foreign
supplier is cheaper, $p_{Ht}(i)>p_{Ft}(i),$ so that the goods are imported,
and $\mathcal{\Psi }_{t}$ denotes the set of products in which the domestic
supplier is cheaper: $p_{Ht}(i)\leq p_{Ft}(i)$.

\subsubsection{Real GDP}

What does real GDP look like in this setting? It would be tempting to
replace current-period prices in nominal GDP with base-period prices. This
direct-valuation method is how Kehohe and Ruhl (2008) construct real values
in their one-good economy, but this is not how NIPA series are constructed.
In NIPA, nominal quantities are deflated by \emph{price indices} to obtain
real quantities. More specifically, real GDP is obtained as: 
\begin{equation}
y_{t}^{\func{real}}=\underset{\text{real domestic component}}{\underbrace{%
\frac{1}{P_{Ht}}\left[ p_{t}^{y}q_{t}-\sum\limits_{i\in \mathcal{\Psi }%
_{t}}p_{Ht}(i)m_{Ht}(i)\right] }}-\underset{\text{real imports}}{\underbrace{%
\frac{1}{P_{Ft}}\sum\limits_{i\in \mathcal{M}_{t}}p_{Ft}(i)m_{Ft}(i)}},
\label{real}
\end{equation}%
where $\left[ p_{t}^{y}q_{t}-\sum\limits_{i\in \mathcal{\Psi }%
_{t}}p_{Ht}(i)m_{Ht}(i)\right] $ is nominal gross output minus the value of
domestic inputs; $\sum\limits_{i\in \mathcal{M}_{t}}p_{Ft}(i)m_{Ft}(i)$ is
the nominal value of imports; $P_{Ht}$ is the (domestic price index based on
a Fisher formula) and $P_{Ft}$ is the import price index (also based on the
Fisher formula). Price indexes are prepared by the Bureau of Labour
Statistics, which deliberately does not mix up import prices with domestic
prices.\footnote{%
We could assume a different basket for exports and domestic sales, but to
keep matters simple, we maintain the assumption of a single final-use good.}

To understand how terms-of-trade affect measured productivity, note first
that nominal GDP in (\ref{nominal}) increases when the price of foreign
inputs $p_{Ft}(i)$ decrease, all else equal. The question we want to address
now is whether real GDP also increases, once the\ price-index adjustment is
carried out.

Price indexes in general, and $P_{Ft}$ in particular, only measure a subset
of goods, and, more importantly, a \emph{non-random} subset of goods. To be
included in the current price index, a product had to be transacted both in
the base period, as well as in the current period. Intuitively, this biases
the import price index $P_{Ft}$ in (\ref{real}) towards \emph{no change}.
This bias in the import price index is known in the pass-through literature
as the product-replacement bias (see for example Nakamura and Steinsson,
2012). With $P_{Ft}$ relatively unchanged, real GDP is clearly influenced by
the change in current import prices, $p_{Ft}(i)^{\prime }s.$

To illustrate the idea, consider a drop in the foreing price of a given
product $i,$ previously sourced from domestic suppliers. As the foreign
price drops, domestic expenditures switch to the foreign supplier. The
product is hence dropped from the domestic price index $P_{Ht}$ but it does
not enter in the import price index $P_{Ft},$ as this is a new entry for
that product in the index basket. (Recall that two price entries of the same
product are needed for a given product to make it to the index.) Hence,
effectively, the decline in the input price is not recorded in the import
price index $P_{Ft}$ (or the domestic index $P_{Ht}$ for that matter). $%
P_{Ft}$ remains unchanged, and as a consequences $y_{t}^{\func{real}}$
increases with the fall in product $i$'s price, as nominal value added
increases when input expenditures switch to the lower-cost supplier abroad.
(For simplicity, and to keep Kehoe-Ruhl first-order approximation exercise,
we assume here that the actual quantity of the input used in production does
not change, only the supplier changes, but the argument naturally goes
through with quantity adjustments). In what follows, we explain in more
detail the bias towards no change in $P_{Ft}$ and how in the Eaton-Kortum
mechanism, this bias implies that increases in productivity abroad (declines
in import prices), lead to domestic real productivity gains.

\subsubsection{What does the Bureau of Labor Statistics (BLS) measure?}

The price indexes produced by the BLS are weighted averages of price changes
for goods or services that are in the sample in both periods. They are based
on the Fisher formula. The indexes produced cover different components of
output; the import price index in particular is computed over the set of
goods imported in both the current and base period. Calling $0$ the base
period and $t$ the curent period, the actual formula is a weighted average
between the Laspeyres and the Paasche indexes: 
\begin{equation*}
P_{Ft}\equiv \left[ \frac{\sum\limits_{i\in \mathcal{M}_{0}\cup \mathcal{M}%
_{t}}p_{Ft}(i)m_{F0}(i)}{\sum\limits_{i\in \mathcal{M}_{0}\cup \mathcal{M}%
_{t}}p_{F0}(i)m_{F0}(i)di}\right] ^{0.5}\left[ \frac{\sum\limits_{i\in 
\mathcal{M}_{0}\cup \mathcal{M}_{t}}p_{Ft}(i)m_{Ft}(i)}{\sum\limits_{i\in 
\mathcal{M}_{0}\cup \mathcal{M}_{t}}p_{F0}(i)m_{Ft}(i)di}\right] ^{0.5}
\end{equation*}%
As stressed earlier, BLS compares current import prices to the past prices
of the \emph{same good} from the \emph{same supplier}. If a firm switches
from a Mexican to a Chinese supplier, because the latter is cheaper, the BLS
will miss this price change in the index. Moreover, the BLS certainly does
not mix up import prices with domestic prices: it creates separate price
indices. So if the substitution is from an American supplier to a Chinese
supplier, the BLS's index will certainly miss this change in price. (There
is even a risk that a change within Chinese suppliers due to a price change
gets lost in translation.) This implies that both in practice as well as in
models in which price changes lead to a change in suppliers terms of trade
changes will affect measured real GDP.

In the EK model in particular, the switch in suppliers is at the heart of
the productivity gains from trade. In such model, the BLS procedure will not
record any of the price changes in its index, which will remain unchanged,
hence leading to measured productivity gains and increases in real GDP. Put
differently, the Kehoe--Ruhl's argument does not apply in this setting. The
intuition should now be clear. Suppose Chinese goods have become 10 percent
cheaper. Due to the winner-take-all nature of Ricardian competition, the
Chinese will be competitive in a wider range of goods. These goods are off
the radar from the BLS---they have been switched from an American or
higher-cost foreign supplier to Chinese firms. The goods that the BLS \emph{%
does} measure are not a random sample from all goods: they are the ones in
which China is still relatively expensive (otherwise, a switch would have
occurred). The BLS index will hence miss the price change and real value
added will be directly affected by the decline in Chinese input prices.

To sum up, the Eaton-Kortum mechanism can indeed account for the empirical
association between terms of trade and productivity in the data. Indeed,
there is no is no puzzle: changes in measured real GDP capture the foreign
gains in productivity transmitted through lower import prices.\footnote{%
While economists may view this result favourably---gains from trade should
be somehow reflected in real GDP!---it has generated some degree of
discomfort in other cirles. In particular, a series of articles in the
Business Week Review have lamented the fact that real GDP is reflecting
\textquotedblleft ghost\textquotedblright\ gains in productivity---gains
coming from other countries---due to substitution towards lower-cost
suppliers abroad.}

\bigskip 

\bigskip 

First order appoximation with multiple goods:$m_{t}=\left\{
\sum_{0}^{1}[m_{Ft}(i)+m_{Ht}(i)]^{\alpha }di\right\} ^{1/\alpha }$

\bigskip 

\begin{equation}
q_{t}=f(l,m_{t})
\end{equation}%
\begin{equation}
y_{t}^{\func{real}}=\underset{\text{real domestic component}}{\underbrace{%
\frac{1}{P_{Ht}}\left[ p_{t}^{y}q_{t}-\sum\limits_{i\in \mathcal{\Psi }%
_{t}}p_{Ht}(i)m_{Ht}(i)\right] }}-\underset{\text{real imports}}{\underbrace{%
\frac{1}{P_{Ft}}\sum\limits_{i\in \mathcal{M}_{t}}p_{Ft}(i)m_{Ft}(i)}},
\end{equation}

\end{document}
