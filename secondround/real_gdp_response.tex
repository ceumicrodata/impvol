We defined real GDP in the model as value-added ($w_{t}L_{t}$ in the paper's
notation) deflated by the welfare-based price index ($P_{t})$. This is the
welfare-relevant measure of real GDP. In the data, we measure real GDP as
nominal value-added (in local currency) divided by the consumer price index
(CPI), which is an expenditure-based Laspeyres price index. It can be easily
shown that the change in the welfare-based price index in the model, to a
first-order Taylor approximation, is equal to the change in a Laspeyres
price index. In other words, there is a very close mapping between measured
real value-added changes in the data and in the model.

To be clear, we focus on the nominal value-added deflated by the CPI because
it maps closely and transparently into the welfare-relevant measure of real
GDP dictated by the model. Burstein and Cravino (2015) make this point very clearly, that a CPI-deflated income provides a good approximation to welfare in a whole class of trade models.
Hence, deliberately we stay away from the real
GDP measure on which Kehoe and Ruhl (2008) focus. The latter measure has no
clear relation with welfare, and, from a practical point of view, it seems difficult to reconcile to what the BLS specifically does. Kehoe and Ruhl (2008)
assume direct valuation, while in practice the BLS uses a very complicated
(and opaque) combination of double deflation, extrapolation, direct
valuation and rules for basket replacement that would be nearly impossible
to replicate following their instruction manuals. The problem gets
compounded when trying to understand the procedure of other national
statistical offices. In contrast, changes in expenditure-based indexes like
the CPI are in this regard more transparent, and hence preferable.

We have modified the text so that the new version of the manuscript makes
the mapping between data and model-generated volatility clear.

