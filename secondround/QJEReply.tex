%2multibyte Version: 5.50.0.2953 CodePage: 65001

\documentclass[12pt]{article}
%%%%%%%%%%%%%%%%%%%%%%%%%%%%%%%%%%%%%%%%%%%%%%%%%%%%%%%%%%%%%%%%%%%%%%%%%%%%%%%%%%%%%%%%%%%%%%%%%%%%%%%%%%%%%%%%%%%%%%%%%%%%%%%%%%%%%%%%%%%%%%%%%%%%%%%%%%%%%%%%%%%%%%%%%%%%%%%%%%%%%%%%%%%%%%%%%%%%%%%%%%%%%%%%%%%%%%%%%%%%%%%%%%%%%%%%%%%%%%%%%%%%%%%%%%%%
\usepackage{amssymb}
\usepackage{lettopti,sw20let1}
\usepackage{setspace}

%TCIDATA{OutputFilter=LATEX.DLL}
%TCIDATA{Version=5.50.0.2953}
%TCIDATA{Codepage=65001}
%TCIDATA{<META NAME="SaveForMode" CONTENT="1">}
%TCIDATA{BibliographyScheme=Manual}
%TCIDATA{Created=Monday, October 04, 2004 07:21:58}
%TCIDATA{LastRevised=Tuesday, November 29, 2016 17:48:43}
%TCIDATA{<META NAME="GraphicsSave" CONTENT="32">}
%TCIDATA{<META NAME="DocumentShell" CONTENT="Other Documents\SW\Letter - SW Letter #1">}
%TCIDATA{Language=British English}
%TCIDATA{CSTFile=tcilett.cst}

\newtheorem{theorem}{Theorem}
\newtheorem{acknowledgement}[theorem]{Acknowledgement}
\newtheorem{algorithm}[theorem]{Algorithm}
\newtheorem{axiom}[theorem]{Axiom}
\newtheorem{case}[theorem]{Case}
\newtheorem{claim}[theorem]{Claim}
\newtheorem{conclusion}[theorem]{Conclusion}
\newtheorem{condition}[theorem]{Condition}
\newtheorem{conjecture}[theorem]{Conjecture}
\newtheorem{corollary}[theorem]{Corollary}
\newtheorem{criterion}[theorem]{Criterion}
\newtheorem{definition}[theorem]{Definition}
\newtheorem{example}[theorem]{Example}
\newtheorem{exercise}[theorem]{Exercise}
\newtheorem{lemma}[theorem]{Lemma}
\newtheorem{notation}[theorem]{Notation}
\newtheorem{problem}[theorem]{Problem}
\newtheorem{proposition}[theorem]{Proposition}
\newtheorem{remark}[theorem]{Remark}
\newtheorem{solution}[theorem]{Solution}
\newtheorem{summary}[theorem]{Summary}
\newenvironment{proof}[1][Proof]{\noindent\textbf{#1.} }{\ \rule{0.5em}{0.5em}}
\newfield{letterfromtele}{From phone: }
\newfield{letterfromloca}{From location: }
\newfield{letterretaddr}{Return address: }
\input{tcilatex}
\begin{document}


\begin{letterfromaddr}
Silvana Tenreyro

London School of Economics

Department of Economics

Houghton St.

London WC2A 2AE

\bigskip
\end{letterfromaddr}

\begin{letterto}
Professor Robert Barro
\end{letterto}

\begin{lettertoaddr}
Quarterly Journal of Economics

Editorial Office
\end{lettertoaddr}

\begin{letterdate}
December 3, 2016\bigskip
\end{letterdate}

\begin{lettersubj}
MS \textquotedblleft Diversification through Trade\textquotedblright \bigskip
\end{lettersubj}

\begin{letteropening}
Dear Robert:
\end{letteropening}

\setstretch{1.5}We thank you, Pol, and the referees very much for the
detailed and helpful comments on the manuscript \textquotedblleft
Diversification through Trade.\textquotedblright\ We are enclosing a revised
version of the manuscript that addresses the issues raised in your letter
and in the referee reports. Attached is also a Supplemental Appendix, which
contains the mathematical derivations and\ calibration details.

In your letter, you kindly asked us to revise the paper along five
dimensions: (1) clarifying the connection between value-added changes in the
model and in the data; (2) relaxing the assumption on the immobility of
equipped labour; (3) allowing for input-output linkages and current account
deficits; (4) reorganizing the writing of the paper, with a new Appendix
for\ part of the more technical material; and (5) changing the elasticity of
substitution across varieties. We have followed your advice and are happy
with the changes, which we think have led to a better paper. We have
furthermore addressed the questions and suggestions raised in the individual
referee reports, as we explain in the enclosed responses to the referees.

In what follows, we describe each of the issues mentioned in your letter and
the steps taken to address them in this revised version.\bigskip

\textbf{1. Relation between measured GDP changes in the data and model}

We agree that this issue was not clearly addressed in the previous version
of the manuscript. The new version clearly explains the measurement of real
GDP in the model, which has a straighforward correspondence to that in the
data and quantitative exercise.

We defined real GDP in the model as value-added ($w_{t}L_{t}$ in the paper's
notation) deflated by the welfare-based price index ($P_{t})$. This is the
welfare-relevant measure of real GDP. In the data, we measure real GDP as
nominal value-added (in local currency) divided by the consumer price index
(CPI), which is an expenditure-based Laspeyres price index. It can be easily
shown that the change in the welfare-based price index in the model, to a
first-order Taylor approximation, is equal to the change in a Laspeyres
price index. In other words, there is a very close mapping between measured
real value-added changes in the data and in the model.

To be clear, we focus on the nominal value-added deflated by the CPI because
it maps closely and transparently into the welfare-relevant measure of real
GDP dictated by the model. Burstein and Cravino (2015) make this point very clearly, that a CPI-deflated income provides a good approximation to welfare in a whole class of trade models.
Hence, deliberately we stay away from the real
GDP measure on which Kehoe and Ruhl (2008) focus. The latter measure has no
clear relation with welfare, and, from a practical point of view, it seems difficult to reconcile to what the BLS specifically does. Kehoe and Ruhl (2008)
assume direct valuation, while in practice the BLS uses a very complicated
(and opaque) combination of double deflation, extrapolation, direct
valuation and rules for basket replacement that would be nearly impossible
to replicate following their instruction manuals. The problem gets
compounded when trying to understand the procedure of other national
statistical offices. In contrast, changes in expenditure-based indexes like
the CPI are in this regard more transparent, and hence preferable.

We have modified the text so that the new version of the manuscript makes
the mapping between data and model-generated volatility clear.


\bigskip

\textbf{2. Labour Mobility}

Following Pol's and the referees' suggestions, the new manuscript studies
the sensitivity of the results to the assumption of labour immobility across
sectors within a period. In Section V we extend the model by introducing
sectoral reallocation costs and quantitatively study their role. As Pol
conjectured, based on our paper-and-pencil results for the one-sector model,
the main qualitative results of the model remain unchanged. Moreoever, the
quantitative results do not appear to be hugely sensitive when we calibrate
the adjustment cost's parameter to match sectoral mobility data from EU
KLEMS.

\bigskip

\textbf{3. Input-output linkages and trade imbalances}

The new version of the paper allows for input-output linkages and trade
imbalances in the baseline calibration. The results are summarized in Table
1. We close each channel in turn to assess their bearing on the results. The
outcomes from these exercises are shown in Tables 3 and 4. When we compare
the results from the baseline case in Table 1 (detailed input-output matrix)
with our original specification in Table 3 (simple input-output linkages),
the main patterns are unchanged and the quantitative outcomes are remarkably
close for most countries. To understand the similarity in results, recall
that our original model already allowed for input-output linkages, though
admittedly a simpler fashion. In light of the results, we conclude the
simpler original structure appears to be a very good approximation to the
more complex input-output structure that we now use as baseline calibration.

Similarly,\ closing down trade imbalances does not change the sign of the
effects and indeed, for most countries the quantitative results with and
without trade imbalances are very close.\bigskip

\textbf{4. Re-writing}

\bigskip Following Pol's suggestions, we have reorganized the writing of the
paper and moved most technical derivations to an Appendix.

\textbf{5. The value of }$\eta $\textbf{\ }

As suggested by Pol, we have done all the quantitative exercises under the
calibrated value of $\eta =4$, in line with Broda and Weinstein (2006)'s
median value for this elasticity.\bigskip

We hope you will find the changes to the paper satisfactory. We thank you
very much again for your comments and suggestions. With best regards,\bigskip

Francesco Caselli, Miklos Koren, Milan Lisicky, and Silvana Tenreyro

encl: New Manuscript, Reply to Referees (3)

\begin{center}
\pagebreak

\thispagestyle{plain}\setcounter{page}{1}Reply to Referee 1's comments on

{\Large Diversification through Trade}

by Francesco Caselli, Miklos Koren, Milan Lisicky and Silvana
Tenreyro\medskip

\bigskip \bigskip
\end{center}

We thank the Referee for his/her insightful comments and suggestions, which
we have tried to incorporate in the revised version of the paper. Below is a
reply to the points raised in the report and the steps we took to address
them. (We reproduce the referee report's headings for convenience.)\bigskip

\textbf{1. Methodology: }

Following the referee's suggestion, the new manuscript has added to the
literature review other papers that use quantitative models of trade to
address a variety of questions in Development or International Economics. In
particular, the paper now cites Costinot, Donaldson and Komunjer (2012),
Levchenko and Zhang (2013), Levchenko and Zhang (2014), di Giovanni,
Levchenko and Zhang (2014), and Allen,  Arkolakis, Takahashi (2014. The
difference with these papers lies in the\ actual research question, not in
the method (alhough there are differences in the implementation and
modelling assumptions that would be apparent to readers comparing the
various papers to ours).\bigskip 

\textbf{2. Adjustment costs:}

Following the referee's advice, we now study the sensitivity of our results
to the assumption of labour immobility across sectors within a period. In
Section V we extend the model by introducing ex post sectoral reallocation
of equipped labour. We assume this can be done at a cost. We calibrate the
adjustment cost parameter to match data on sectoral mobility from EU Klems.
The results, displayed in Table 5, appear robust to this modification.

Regarding sectoral complementarities, we allowed for a more complex
production structure with input-output linkages in line with the data. This
did not lead to material differences in the quantitative results. 

The most important parameter in the model in terms of sensitivity is $\theta 
$, the parameter governing the scope for comparative advantage. We provide
results for a wide parametric range so that readers can asses the effects
depending on their priors.\bigskip 

\textbf{3. Stochastic process}

We think the new version of the paper clarifies this point. We feed in the
model the stochastic processes inferred from the data. There is an
anticipated component of productivity embedded in the trend, which certainly
plays a quantitatively important role in firms' allocation decisions. The
shocks in the paper are the (unanticipated) deviations from trends.\bigskip 

\textbf{4. Disaggregation}

As the referee points out, the main constraint to further disaggragation is
data availability, particularly for less developed countries. Twenty two
sectors is the best we can do without sacrificing the number of countries in
the analysis. We are fully aware that the gain in number of countries comes
at a loss in terms of aggregation. Taking a more granular approach will
certainly be helpful in many instances, but in this context, it would entail
focusing on one---or two---countries. We discuss this point in the paper,
and cite work by di Giovanni and Levchenko (2012) and di Giovanni, Levchenko
and Mejean (2015) for readers interested in firm-level data.\bigskip 

\textbf{5. Counterfactuals}

We have followed the referee's advice and expanded the counterfactual
analysis, studying the role of different variables, one at a time. In
particular, in Section V of the new manuscript, we carry out the analysis
with and without labour mobility across sectors, with and without
input-output linkages, with and without trade imbalances. The results are
summarized in Tables 1 through 5.

We retained the focus on the question we want to address, namely the effect
of trade costs (which in principle could be influenced by policy) on
volatility. Regarding one of the possibilities discussed by the referee, we
opted for keeping the trade costs constant (as opposed to keeping the trade
volumes constant) in line with our motivation, as trade volumes are not deep
parameters and hence a counterfactual that artificially kept trade volumes
constant would be subject to well-known objections based on the Lucas'
critique. The framework is very flexible and we will make our programs
available if readers wish to perform additional experiments.\bigskip 

\textbf{6. Welfare}

The focus of the paper is on the positive effects of trade openness. If
volatility decreases with trade openness, as it does in the majority of
cases in our quantitative analysis, trade openness becomes a win-win option.
Theoretically, volatility could increase and of course in extremes in which
this increase is large and/or a country puts a high value on stability,
welfare may actually decline. In the cases in which we observe increases in
volatility due to trade openness in our quantitative analysis, these
increases tend to be small, suggesting that the increases in productivity
would still dominate the tradeoff and trade would be welfare enhancing for
the representative agent.\bigskip

\textbf{7. Measurement}

We defined real GDP in the model as value-added ($w_{t}L_{t}$ in the paper's
notation) deflated by the welfare-based price index ($P_{t})$. This is the
welfare-relevant measure of real GDP. In the data, we measure real GDP as
nominal value-added (in local currency) divided by the consumer price index
(CPI), which is an expenditure-based Laspeyres price index. It can be easily
shown that the change in the welfare-based price index in the model, to a
first-order Taylor approximation, is equal to the change in a Laspeyres
price index. In other words, there is a very close mapping between measured
real value-added changes in the data and in the model.

To be clear, we focus on the nominal value-added deflated by the CPI because
it maps closely and transparently into the welfare-relevant measure of real
GDP dictated by the model. Burstein and Cravino (2015) make this point very clearly, that a CPI-deflated income provides a good approximation to welfare in a whole class of trade models.
Hence, deliberately we stay away from the real
GDP measure on which Kehoe and Ruhl (2008) focus. The latter measure has no
clear relation with welfare, and, from a practical point of view, it seems difficult to reconcile to what the BLS specifically does. Kehoe and Ruhl (2008)
assume direct valuation, while in practice the BLS uses a very complicated
(and opaque) combination of double deflation, extrapolation, direct
valuation and rules for basket replacement that would be nearly impossible
to replicate following their instruction manuals. The problem gets
compounded when trying to understand the procedure of other national
statistical offices. In contrast, changes in expenditure-based indexes like
the CPI are in this regard more transparent, and hence preferable.

We have modified the text so that the new version of the manuscript makes
the mapping between data and model-generated volatility clear.


\bigskip 

\textbf{8. Other minor comments}

i) Following the referee's suggestion, we now comment on the inferred values
of trade costs in the text.

ii) We have\ now merged the two sections in the appendix.

iii) We have altered the order of the columns in the footnote, as suggested.
We believe the new footnote should be clear to readers.\bigskip

Once again, thank you very much for your constructive and insightful
comments, which have helped improving the paper.\medskip \medskip \bigskip

Francesco Caselli, Miklos Koren, Milan Lisicky, and Silvana Tenreyro

\thispagestyle{plain}\pagebreak

\begin{center}
\thispagestyle{plain}\setcounter{page}{1}Reply to Referee 2's comments on

{\Large Diversification through Trade}

by Francesco Caselli, Miklos Koren, Milan Lisicky and Silvana
Tenreyro\medskip

\bigskip
\end{center}

We thank the Referee for his/her insightful comments and suggestions, which
we have tried to incorporate in the revised version of the paper. Below is a
reply to the points raised in the report and the steps we took to address
them. (For convenience, we reproduce in italics the referee's first sentence
in each of the comments.)\bigskip

1)\textit{\ \textquotedblleft Donna Costello took a look at a very closely
related question in a 1993 (uncited) JPE article titled `A Cross Country,
Cross Industry Comparison of Productivity Growth'.\textquotedblright }

We found the reference to Costello (1993)'s work relevant for our work and
thank the referee for referring us to her work. Costello (1993) empirically
studies the Sollow residuals of five industrialized countries, finding that
nation-specific shocks are quantitatively more important than
industry-specific shocks. Our work explores, through a different approach
and a broader set of countries, what happens when countries open up to
international trade. When nation-specific shocks are prevalent, as in
Costello (1993)'s sample and many of the countries in our sample, volatility
declines as the diversification of shocks outweighs the effects of sectoral
specialization.\bigskip

2)\textit{\ \textquotedblleft First, I think it would help if the authors
could cast their procedure for identifying separately the role of bilateral
trade costs and national and sector productivities in terms of the language
developed in the paper `Universal Gravity'.\textquotedblright }

Following the Referee's suggestion, we now relate our work to Allen,
Arkolakis and Takahashi (2014). In terms of the language developed in their
paper \textquotedblleft Universal Gravity,\textquotedblright\ our
multi-sector model is underidentified, in the sense that (sectoral) trade
flows alone are not sufficient to identify what the authors call
\textquotedblleft gravity constants.\textquotedblright\ This is why we need
to resort to additional assumptions (e.g., symetric trading costs) and
sources of data, in particular, sectoral prices,\ in order to identify
sectoral productivity changes. \bigskip

3) \textit{\textquotedblleft Second, I would like to see a better resolution
of the issues surrounding the measurement of GDP.\textquotedblright }

We have clarified the relation with Burstein and Cravino (2015), as
suggested by the Referee. We agree that this issue was not clearly addressed
in the previous version of the manuscript. The new version clearly explains
the measurement of real GDP in the model, which has a straighforward
correspondence to that in the data and quantitative exercise.

We defined real GDP in the model as value-added ($w_{t}L_{t}$ in the paper's
notation) deflated by the welfare-based price index ($P_{t})$. This is the
welfare-relevant measure of real GDP. In the data, we measure real GDP as
nominal value-added (in local currency) divided by the consumer price index
(CPI), which is an expenditure-based Laspeyres price index. It can be easily
shown that the change in the welfare-based price index in the model, to a
first-order Taylor approximation, is equal to the change in a Laspeyres
price index. In other words, there is a very close mapping between measured
real value-added changes in the data and in the model.

To be clear, we focus on the nominal value-added deflated by the CPI because
it maps closely and transparently into the welfare-relevant measure of real
GDP dictated by the model. Burstein and Cravino (2015) make this point very clearly, that a CPI-deflated income provides a good approximation to welfare in a whole class of trade models.
Hence, deliberately we stay away from the real
GDP measure on which Kehoe and Ruhl (2008) focus. The latter measure has no
clear relation with welfare, and, from a practical point of view, it seems difficult to reconcile to what the BLS specifically does. Kehoe and Ruhl (2008)
assume direct valuation, while in practice the BLS uses a very complicated
(and opaque) combination of double deflation, extrapolation, direct
valuation and rules for basket replacement that would be nearly impossible
to replicate following their instruction manuals. The problem gets
compounded when trying to understand the procedure of other national
statistical offices. In contrast, changes in expenditure-based indexes like
the CPI are in this regard more transparent, and hence preferable.

We have modified the text so that the new version of the manuscript makes
the mapping between data and model-generated volatility clear.


\bigskip

Once again, thank you very much for your constructive and insightful
comments, which have helped improving the paper.\medskip \medskip \bigskip

Francesco Caselli, Miklos Koren, Milan Lisicky, and Silvana
Tenreyro\pagebreak

\begin{center}
\thispagestyle{plain}\setcounter{page}{1}

Reply to Referee 3's comments on

{\Large Diversification through Trade}

by Francesco Caselli, Miklos Koren, Milan Lisicky and Silvana
Tenreyro\medskip \medskip \bigskip
\end{center}

We thank the Referee for his/her insightful comments and suggestions, which
we have tried to incorporate in the revised version of the paper. Below is a
reply to the points raised in the report and the steps we took to address
them.

\bigskip

1) Input-output linkages

Following the suggestion made by the Referee, the new version of the paper
incorporates input-output linkages in the baseline model. The results are
summarized in Table 1. We close the input-output linkages as explained in
Section V to assess their bearing on the results. The outcomes from this
exercise are shown in Tables 3. When we compare the results from the
baseline case in Table 1 (detailed input-output matrix) with our original
specification in Table 3 (simple input-output linkages), the main
qualitative patterns are unchanged and the quantitative outcomes are
remarkably close for most countries. To understand the robustness of the
results, note that our original model already allowed for input-output
linkages, though admittedly a simpler fashion. In light of the results, we
conclude that the simpler original structure appears to be a very good
approximation to the more complex input-output structure that we now use as
baseline calibration.\bigskip 

2) Trade imbalances

We thank the referee for bringing up this issue, which we now discuss in
more detail in the paper. Section V in the new version of the paper now
makes clear the role that trade imbalances play in the quantitative effects
of trade opennes on volatility. We have not endogenized the trade
imbalances, however. The model is already very complex and bringing in a
realistic treatment of sovereign debt, lending, and defaults would be beyond
our programming capabilities. 

The baseline results in Table 1 incorporate trade imbalances, calibrated to
match the data. Table 3 presents the results when those imbalances are set
to zero.\bigskip 

3) Robustness checks I

There is no strong a priori reason to expect the backed-out sectoral
productivities would align with those computed by EU KLEMS. The backed-out
sectoral productivities in our quantitative exercise correspond to the
intrinsic or primitive productivities, after having purged the influence of
international trade. The sectoral productivities measured in the data as a
residual should be in principle different from the primitive productivities
we back out because, with international trade, these residuals will embed
not simply the intrinsic productivity of the sector but also the
productivities of all other countries (and sectors) from which the given
sector buys its inputs. Hence, we could not convince ourselves that this
exercise could act as a test for robustness.\bigskip 

4) Robustness checks II

We believe the new version of the paper provides a number of informative
robustness checks. The effect of trade on volatility will indeed change if
we alter the sample of countries parrticipating in international trade in
the counterfactual.\ (This would be akin to increasing trading costs to
infinity to some countries.) The possibilities for counterfactuals here are
incredibly large, if one takes into account all possible configurations with
different numbers and combination of countries in global trade. We were not
clear whether the referee wanted to see one country or set of countries in
particular being removed from the sample. After discussing the
possibilities, we concluded it would be best to make our matlab programs
available and ready to use for readers interested in exploring other
counterfactuals. But answering the final question of the referee, our
results indicate that international trade has, for most countries, led to a
reduction in volatility, though, as shown, the size of the effect has been
highly heterogeneous.

\medskip \medskip \bigskip

Once again, thank you very much for your constructive and insightful
comments.

Francesco Caselli, Miklos Koren, Milan Lisicky, and Silvana Tenreyro

\pagebreak

\begin{center}
\thispagestyle{plain}\setcounter{page}{1}

Reply to Referee 4's comments on

{\Large Diversification through Trade}

by Francesco Caselli, Miklos Koren, Milan Lisicky and Silvana
Tenreyro\medskip \medskip \bigskip
\end{center}

We thank the Referee for his/her insightful comments and suggestions, which
we have tried to incorporate in the revised version of the paper. Below is a
reply to the points raised in the report and the steps we took to address
them.

\bigskip

1) Specialization and Volatility

The Referee correctly points out the trade offs present in our paper. The
role of diversification indeed can be seen clearly in the pencil-and-paper
one-sector version of our model and the effect of sectoral specialization on
volatility is already well understood. Given that one of the editors has
explicitly asked us to cut on material that is already understood and focus
on the main quantitative exercise, we decided not to present the two-country
version of the model in the paper, as requested. However, we connected the
diversification result to the diminishing returns in Acemoglu and Ventura
(2002). 

We have also follow the Referee's advice regarding our discussion of Kehoe
and Ruhl (2008), which we agree was confusing. More specifically, measurment
was not clearly addressed in the previous version of the manuscript. The new
version clearly explains the measurement of real GDP in the model, which has
a straighforward correspondence to that in the data and quantitative
exercise.

Real GDP in the model is defined as value-added ($w_{t}L_{t}$ in the paper's
notation) deflated by the welfare-based price index ($P_{t})$. This is the
welfare-relevant measure of real GDP. In the data, real GDP is measured as
nominal value-added (in local currency) divided by the consumer price index
(CPI), which is an expenditure-based Laspeyres price index. It can be easily
shown that the change in the welfare-based price index in the model, to a
first-order Taylor approximation, is equal to the change in a Laspeyres
price index. In other words, there is a very close mapping between measured
real value-added\ changes in the data\ and in the model.

To be clear, we focus on the nominal value-added deflated by the CPI because
it maps closely and transparently into the measure of real GDP dictated by
the model. Hence, deliberately we stay away from the real GDP measure on
which Kehoe and Ruhl (2008) focus. The latter measure has no clear relation
with welfare, and, from a practical point of view, it does not correspond
either to what the BLS actually does. Kehoe and Ruhl (2008) assume direct
valuation, while in practice the BLS uses a very complicated (and opaque)
combination of double deflation, extrapolation, direct valuation and rules
for basket replacement that would be nearly impossible to replicate
following their instruction manuals. The problem gets compounded when trying
to understand the procedure of other national statistical offices. In
contrast, changes in expenditure-based indexes like the CPI are in this
regard more transparent, and hence preferable.

We have modified the text so that the new version of the manuscript makes
the mapping between data and model-generated volatility clear.

\bigskip

2) The Extensive Margin of Trade

This point is related to the comment above, and stems in part from our
confusing discussion of Kehoe and Ruhl in the previous version of the paper.
(We apologize for the confusion.) Since we measure real income as value
added deflated by an expenditure-based price index both in the model and in
the data, changes in productivity at home or abroad will be reflected in
real income even if they come entirely from the intensive margin of trade.
To give a concrete example, suppose the United States consumes Colombian
coffee and Colombia sees an aggregate (or sectoral) increase in
productivity, which leads to a decline in the price of coffee. The decline
in the retail price of coffee\ in the US will be reflected in the US CPI and
contribute to higher real income in the US. In other words, our mechanism
also operates through the intensive margin.

We have tried to make this point clear in the new version of the paper.

\bigskip

3) The One-Period Immobility of Labour

We have followed Referee's advice and studied an extension of the model
which allows for sectoral reallocation of equipped labour. We assume
reallocation is costly and calibrate the ajdustment cost parameter to match
sectoral mobilita data from EU KLEMS. The results are shown and discussed in
Table 5 and Section V of the new manuscript.

\bigskip

4) Other Points/Comments

-We think the new version of the paper provides a number of informative
robustness checks that help understand what drives the results. A parameter
playing a key role in the results is $\theta $, which measures the scope of
comparative advantage. When there is significant scope for comparative
advantage, reductions in trade costs can significantly reduce volatility. We
also learnt\ (or confirmed) that the relative quantitative importance of
sectoral versus aggregate shocks is critical for the results. Sectoral
shocks in the data are not as prevalent when compared with aggregate shocks,
and hence the diversification channel is quantitatively more important than
the specialization channel.

-We agree with the Referee's comment on Backus-Kehoe-Kydland. Our point is
that, in practice, consumption and output move in the same direction most of
the time (in fact consumption volatility and output volatility have an
extremely high correlation). Hence, focusing on GDP volatility is a good
proxy for consumption volatility.

-The time dimension becomes important when it comes to equipped labour's
adjustment \ A two-period version could suffice, but we think a
multiple-period version makes it clear that there is also a trend-component
playing a role in the quantitative calibration.

-At the calibration stage, since we need to back out the productivity
series, we need to calibrate the labour shares, which, ex ante, equal the
trend components. (Nothing changes if we calibrate this using the actual
shares.)

\medskip \medskip \bigskip

Once again, thank you very much for your constructive and insightful
comments.

Francesco Caselli, Miklos Koren, Milan Lisicky, and Silvana Tenreyro

\end{document}
